\documentclass[11pt]{report}
\usepackage{amsmath}
\usepackage{amssymb}
%\usepackage{bbm}
\usepackage{graphicx}
\usepackage{tikz}


\newcommand{\Laplace}{\mathcal{L}}
\newcommand{\ft}{f(t)}
\newcommand{\ftn}[1]{f(#1)}
\newcommand{\ftp}[1]{f^{#1}(t)}
\newcommand{\Fs}{F(s)}
\newcommand{\Fsp}[1]{F^{#1}(s)}
\newcommand{\LaplaceIntegral}{\int_{0}^{\infty}e^{-st}\ft\text{dt}}

\newcommand{\ubt}[1]{\textbf{\underline{#1}}}
\newcommand{\sps}{\\[0.2cm]}
\newcommand{\spn}[1]{\\[#1cm]}
\newcommand{\refn}[1]{(\ref{#1})}
\newcommand{\refx}[1]{\refn{eq:#1}}
\newcommand{\bt}[1]{\textbf{#1}}
\newcommand{\dsp}{\displaystyle}
\newcommand{\NI}{\noindent}
\newcommand{\real}{ \mathbb{R}}
\newcommand{\mbf}[1]{\mathbf{#1}}
\newcommand{\complex}{\mathbb{C}}
\newcommand{\sprime}{'}
\newcommand{\dprime}{''}
\newcommand{\tprime}{'''}
\newcommand{\sbracket}[1]{\left[#1\right]}
\newcommand{\example}[1]{\section*{\ubt{Example #1}}{~}\spn{-1}}
\newcommand{\examples}{\subsubsection*{Examples}{~}\spn{-1}}
\newcommand{\solution}{\subsubsection{\ubt{Solution}}{~}\spn{-1}}
\newcommand{\eg}{\subsection*{\ubt{Example}}{~}\spn{-1}}
%\newcommand{\real}{\mathbbm{R}}

\renewcommand{\baselinestretch}{1.5}
\renewcommand{\contentsname}{Table of Contents}
\renewcommand{\labelenumi}{\arabic{enumi})}
\renewcommand{\labelenumii}{\alph{enumii})}

%\setlength{\parindent}{1em}


\begin{document}
	
	%%%%%%%%%%%%%%%%%%%FRONT COVER%%%%%%%%%%%%%%%%%%%
	\addcontentsline{toc}{chapter}{TITLE PAGE}
	\clearpage
	\thispagestyle{empty}
	\begin{center}
		\Large \bt{LAPLACE TRANSFORMATION METHOD OF SOLVING SYSTEM OF VOLTERRA INTEGRAL EQUATION}
	\end{center}

	\hspace{7cm}
	
	\begin{center}
		\textbf{\textit{BY}}
	\end{center}
	
	\hspace{5cm}
	
	\begin{center}
		\large \textbf{ISHOLA, ABDULMUIZ ADESHINA
			\\
			17/30GQ029}
	\end{center}
	
	\hspace{9cm}
	
	\begin{center}
		A PROJECT SUBMITTED TO THE DEPARTMENT OF MATHEMATICS, FACULTY OF PHYSICAL SCIENCES, UNIVERSITY OF ILORIN, ILORIN, KWARA STATE, NIGERIA.
	\end{center}

	\hspace{7cm}
	
	\begin{center}
		IN PARTIAL FULFILLMENT OF REQUIREMENTS FOR THE AWARD OF BACHELOR OF SCIENCE (B. Sc.) DEGREE IN MATHEMATICS.
	\end{center}
	\hspace{5cm}
	\\ \\ 
	\begin{center}
		\textbf{NOVEMBER, 2022}
	\end{center}

	\newpage
	\pagenumbering{roman}
	\addcontentsline{toc}{chapter}{CERTIFICATION}
	\section*{\begin{center}\textbf{\Large CERTIFICATION}   \end{center}}
	This is to certify that this project was carried out by \textbf{ISHOLA, Abdulmuiz Adeshina} with Matriculation Number  17/30GQ029 in the Department of Mathematics, Faculty of Physical Sciences, University of Ilorin, Ilorin, Nigeria, for the award of Bachelor of Science (B.Sc.) degree in Mathematics.
	\\
	\\
	................................... \qquad \qquad\qquad\qquad\qquad\qquad...................... \\
	Dr. K.A. Bello~~ \quad\qquad\qquad\qquad\qquad\qquad\qquad\qquad Date\\
	Supervisor\\
	\\
	\\
	\\
	...................................... \qquad\qquad\qquad\qquad\qquad\qquad ......................\\
	Prof. K. Rauf      \qquad\qquad\qquad\qquad\qquad\qquad\qquad\qquad\quad     Date\\
	Head of Department\\
	\\
	\\
	\\
	..................................... \qquad\qquad\qquad\qquad\qquad\qquad .......................\\
	Prof.o \quad\qquad\qquad\qquad\qquad\qquad\qquad\qquad         Date\\
	External Examiner 
	
	\newpage
	%%ACKNOLEDGMENTS%%
	\section*{\begin{center}\textbf{\Large ACKNOWLEDGMENTS}\end{center}}
	\addcontentsline{toc}{chapter}{ACKNOWLEDGMENTS} 					
	Firstly, I will give glory to God for his abundant blessings and guidance throughout my stay in school and for giving strength to face all tasks.\\
	
	\NI An academic pursuit is a challenging task that requires all encouragement, proper guidance, direction and support. To this end I wish to express my sincere and profound gratitude to my level advisor and supervisor, Dr. K.A Bello, who supervised and carefully guided the eventual write up of this project.\\
	
	\NI I also appreciate the immeasurable effort of my lecturers in the departmentwho have taught and share their knowledge to me: Prof. J. A. Gbadeyan, Prof. T. O. Opoola, Prof. O. M. Bamigbola, Prof. O. A. Taiwo, Prof. M. O. Ibrahim, Prof.R.B. Adeniyi, Prof. M. S. Dada, Prof. A. S. Idowu, Prof. O. A.
	Fadipe-Joseph, Dr E.O. Titiloye, Dr. Yidiat O. Aderinto, Dr. Catherine N. Ejieji, Dr. B. M. Yisa, Dr J. U. Abubakar, Dr. Gata N. Bakare, Dr T. O. Olotu, Dr. B. M. Ahmed, Dr Idayat F. Usamot, Dr O. A. Uwaheren, Dr O. Odetunde, Dr. Oyekunle, Dr. Ayinla and all other members of staff of the department of mathematics, who contributed greatly to my academic excellence, obtained during my period of study in the department. May God bless them all.\\
	
	\NI My sincere gratitude to my parents Alhaji M.A Ishola and Alhaja A.M Ishola for their continuous love, moral support, prayer, advice and financial support in all my academic undertakings. May you reap the fruit of your labour (Amin).\\
	
	\NI I also want to express my deep and sincere appreciation to my sisters, Muftiat and Mufeedah for their moral support and companionship.\\
	
	\NI Further appreciation goes to my friends, Sa’ad Muhammed, Adebayo Muhammed, Adejumo Lekan, Muhammed Teslim, Tajudeen Mustapha, and others whom I’m not opportuned to mention their names, thank you all for your constant advice, encouragement, love and making my stay on campus worthwhile.\\
	
	
	\newpage
	%%DEDICATION%%
	\section*{\begin{center}\textbf{\Large DEDICATION}\end{center}}
	\addcontentsline{toc}{chapter}{DEDICATION}
	This work is dedicated to the Glory of God for His infinite mercy and guidance over me.\\
	
	\NI My ever caring and loving father, Alhaji M.A Ishola for his continuous guidance and support.\\
	
	\NI My sweet mother Alhaja A.M Ishola for her care, love, advise and motherly role and to all the people that made this project a great success.
	
	\newpage
	%%ABSTRACT%%
	\section*{\begin{center}\textbf{\Large ABSTRACT}\end{center}}
	\addcontentsline{toc}{chapter}{ABSTRACT}
	This project discusses the numerical solution of the system of Volterra integral equations by using Laplace transform method. The Volterra integral equations is a method that can be used to solve initial value problems and integral equations as well, it transforms linear differential equations into algebraic equations and convolution into multiplication. Laplace is also an integral transform that converts a function of a real variable $(x)$ to a function of a complex variable $(S)$.
	
	\newpage
	%%%%%%%%%%%%%%%%%%%TABLE OF CONTENTS%%%%%%%%%%%%%%%%%%%
	\addcontentsline{toc}{chapter}{TABLE OF CONTENTS}
	\tableofcontents
	
	\newpage
	\pagenumbering{arabic}
	%%%%%%%%%%%%%%%%%%%CHAPTER ONE%%%%%%%%%%%%%%%%%%%
	\chapter{GENERAL INTRODUCTION}
	\section{HISTORICAL BACKGROUND}
	J. Fourier (1768-1830) is the initiator of the theory of integral equations. A term integral equation first suggested by Du Bois-Reymond in 1888. Du Bois-Reymond define an integral equation is understood an equation in which the unknown function occurs under one or more signs of definite integration. Late eighteenth and early ninetieth century Laplace, Fourier, Poisson, Liouville and Abel studies some special type of integral equation. The pioneering systematic investigations goes back to late nineteenth and early twentieth century work of Volterra, Fredholm and Hilbert. In 1887, Volterra published a series of famous papers in which he singled out the notion of a functional and pioneered in the development of a theory of functional in theory of linear integral equation of special type. Fredholm presented the fundamentals of the Fredholm integral equation theory in a paper published in 1903 in the Acta Mathematica. This paper became famous almost overnight and soon took its rightful place among the gems of modern mathematics. Hilbert followed Fredholm’s famous paper with a series of papers in the Nachrichten of the Guttingen Academy. (Subrahamanyam Upadhyay. November 4, 2015) 
	
	\section{Integral equation}
	Integral equation is an equation in which the unknown, say a function of a numerical variable, occurs under an integral. That means a functional equation involving the unknown function under one or more integrals. For example,
	\begin{eqnarray*}
		y(x) = f(x) + 5 \int_{0}^{1} e^{t+x} y(t)dt
	\end{eqnarray*}
		
	\section{Volterra Integral equation}
	Second kind of Linear Volterra integral equation defined by,
	\begin{eqnarray*}
		y(x) = f(x) + \lambda\int_{x_0}^{x} K(t,x)y(t)dt
	\end{eqnarray*}
	where $f(x), K(t,x)$ are known functions and $y(x)$ is the unknown function and $\lambda$ is a numerical parameter. Second kind of non-linear Volterra integral equation defined by 
	\begin{eqnarray*}
		y(x) = f(x) + \lambda\int_{x_0}^{x}K_0(t,x,y(t))dt
	\end{eqnarray*}
	where $K_0(t, x, y(t)) = K(t, x)y(t)$. First kind of Linear Volterra integral equation defined by
	\begin{eqnarray*}
			f(x) + \lambda\int_{x_0}^{x}K(t,x)y(t)dt= 0
	\end{eqnarray*}
	First kind of non - linear Volterra integral equation defined by,
	\begin{eqnarray*}
		f(x) + \lambda\int_{x_0}^{x}K_0(t,x)y(t)dt = 0
	\end{eqnarray*}
	
	\section{Fredholm Integral equation}
	Second kind of Linear Fredholm integral equation defined by,
	\begin{eqnarray*}
		y(x) = f(x) + \lambda\int_{x_0}^{x_1} K(t,x)y(t)dt
	\end{eqnarray*}
	where $f(x),K(t,x)$ are known functions and $y(x)$ is the unknown function and $\lambda$ is a numerical parameter.	Second kind of non-linear Fredholm integral equation defined by 
	\begin{eqnarray*}
		y(x) = f(x) + \lambda\int_{x_0}^{x_1}K_0(t,x,y(t))dt
	\end{eqnarray*}
	where $K_0(t, x, y(t)) = K(t, x)y(t)$. First kind of Linear Fredholm integral equation defined by
	\begin{eqnarray*}
		f(x) + \lambda\int_{x_0}^{x_1}K(t,x)y(t)dt= 0
	\end{eqnarray*}
	First kind of non - linear Fredholm integral equation defined by,
	\begin{eqnarray*}
		f(x) + \lambda\int_{x_0}^{x_1}K_0(t,x)y(t)dt= 0
	\end{eqnarray*}
	
	\NI\bt{Remarks} If $f(x) = 0$, then above defined Volterra and Fredholm Integral equations are called homogeneous type otherwise non homogeneous. Main difference between Volterra and Fredholm Integral equations are range of integration in the integral equation.
	
	\subsection{Classification of Integral Equations}
	\par Integral equations are classified according to three different dichotomies:
	\begin{enumerate}
		\item Limits of Integration
			\begin{enumerate}
				\item \textbf{both fixed}: Fredholm equation
				\item  \textbf{one variable}: Volterra equation
			\end{enumerate}
		\item  Placement of unknown function
			\begin{enumerate}
				\item  \textbf{only inside integral}: first kind
				\item \textbf{both inside and outside integral}: second
			\end{enumerate}
		\item Nature of known function $f$
			\begin{enumerate}
				\item \textbf{identically zero}: homogeneous
				\item \textbf{not identically zero}: inhomogeneous
			\end{enumerate}
	\end{enumerate}

	\NI Integral equations are important in many applications. Problems in which integral equations are encountered including \textbf{radiative transfer}, and the \textbf{oscillation} of a string, membrane, or axle. Oscillation problems may also be solved as \textbf{differential equations}.
	
	\section{The Connection Between Differential and Integral Equations (First-Order)}
	Consider the differential equation (initial value problem)
	\begin{equation}
		y'(x) = f(x,y)\hspace{1cm} y(x_0) = y_0\label{eq:1_1}
	\end{equation}
	By integrating $x_0$ to  $x$, we obtain;
	\begin{eqnarray*}
		\int_{x_0}^{x} y'(t)dt = \int{x_0}^{x} f(t,y)dt
	\end{eqnarray*}
	i.e,
	\begin{equation}
		y(x) = y_0 + \int_{x_0}^{x} f(t,y)dt\label{eq:1_2}
	\end{equation}
	On the other hand, if equation \refx{1_2} holds, we see that $y(x_0) = y_0$, and $y'(x) = f(x,y)$ which implies that equation \refx{1_1} holds! Thus the problems \refx{1_1} and \refx{1_2} are equivalent.\\
	
	\NI In fact, it is possible to formulate many initial and boundary value problems as integral equations and vice versa.
	
	\section{The Connection Between Differential and Integral Equations (Second-Order)}
	Assume that we want to solve the initial value problem
	\begin{gather}
		u"(x) + u(x)q(x) = f(x), x>a\\
		u(a) = u_0,~~~~ u'(a) = u_1\notag
	\end{gather}
	%$$,		\textbf{(1.7.1)}\\
	We integrate the equation from $a$ to $x$ and get
	\begin{eqnarray*}
		u'(x) - u_1 = \int_{a}^{x} [f(y) - q(y)u(y)]dy\label{eq:1_3}
	\end{eqnarray*}
	and another integration yields
	\begin{eqnarray*}
		\int_{a}^{x} u'(s)ds = \int_{a}^{x} u_1ds + \int_{a}^{x}\int_{a}^{x} [f(y) - q(y)u(y)]dyds
	\end{eqnarray*}
	we get;
	\begin{eqnarray*}
		u(x) - u_0 = u_1(x-a) + \int_{a}^{x} [f(y) - q(y)u(y)](x - y)dy
	\end{eqnarray*}
	which we can write as,
	\begin{eqnarray*}
		u(x) &=& u_0 + u_1(x-a) + \int_{a}^{x} f(y)(x - y)dy + \int_{a}^{x} q(y)(y - x)u(y)dy\sps
		&=& F(x) + \int_{a}^{x} k(x,y)u(y)dy
	\end{eqnarray*}
	where,
	\begin{eqnarray*}
		F(x) = u_0 + u_1(x - a) + \int_{a}^{x} f(y)(x - y)dy
	\end{eqnarray*}
	and
	\begin{eqnarray*}
			k(x,y) = q(y)(y - x)
	\end{eqnarray*}
	
	\NI This implies that equation \refx{1_3} can be written as \textbf{Volterra equation}
	\begin{eqnarray}
		u(x) = F(x) + \int_{a}^{x} k(x,y)u(y)dy
	\end{eqnarray}
	
	
		
	%%%%%%%%%%%%%%%%%%%CHAPTER TWO%%%%%%%%%%%%%%%%%%%
	\chapter{LITERATURE REVIEW}
	\section{Volterra Integral Equation}
	In mathematics, the Volterra integral equations are a special type of integral equations. They are divided into two groups referred to as the first and the second kind.\\
	
	\NI A linear Volterra equation of the first kind is;
	\begin{eqnarray*}
		f(t) = \int_{a}^{t} K(t, s) x(s) ds
	\end{eqnarray*}
	where $f$ is a given function and $x$ is an unknown function to be solved for. A linear Volterra equation of the second kind is;
	\begin{eqnarray*}
		x(t) = f(t) + \int_{a}^{t} K(t,s) x(s) ds
	\end{eqnarray*}
	
	\NI In operator theory, and in Fredholm theory, the corresponding operators are called Volterra operators. A useful method to solve such equations, the Adomian decomposition method, is due to George Adomian.\\
	A linear Volterra integral equation is a convolution equation if,
	\begin{eqnarray*}
		x(t) = f(t) + \int_{t_{0}}^{t} K(t - s) x(s) ds
	\end{eqnarray*}
	The function $K$ in the integral is called the kernel. Such equations can be analyzed and solved by means of Laplace transform techniques.\\
	
	\NI The Volterra integral equations were introduced by Vito Volterra and then studied by Traian Lalescu in his 1908 thesis, Sur les équations de Volterra, written under the direction of Émile Picard. In 1911, Lalescu wrote the first book ever on integral equations.\\
	
	\NI Volterra integral equations find application in demography, the study of viscoelastic materials, and in actuarial science through the renewal equation.

	\section{Conversion of Volterra equation of the first kind to the second kind}
	A linear Volterra equation of the first kind can always be reduced to a linear Volterra equation of the second kind, assuming that $K(t,t)\neq{0}$. Taking the derivative of the first kind Volterra equation gives us:
	\begin{eqnarray*}
		\frac{df}{dt} = \int_{a}^{t} \frac{\partial K}{\partial t} x(s) ds
	\end{eqnarray*}
	Dividing through by $K(t,t)$ yields:
	\begin{eqnarray*}
		x(t) = \frac{1}{K(t,t)} \frac{df}{dt} - \int_{a}^{t} \frac{1}{K(t,t} \frac{\partial K}{\partial t} x(s) ds
	\end{eqnarray*}
	Defining $\dsp\overline{f}(t) = \frac{1}{K(t,t} \frac{df}{dt}$ and $\dsp\overline{K}(t,s) = -\frac{1}{K(t,t)} \frac{\partial K}{\partial t}$ completes the transformation of the first kind equation into a linear Volterra equation of the second kind.

	\section{Volterra-Fredholm Integral Equations}
	The Volterra-Fredholm integral equations arise from parabolic boundary value problems, from the mathematical modelling of the spatio-temporal development of an epidemic, and from various physical and biological models. The Volterra-Fredholm integral equations appear in the literature in two forms, namely;
	\begin{eqnarray}
		u(x) = f(x) + \lambda_{1} \int_{a}^{x} K_{1}(x,t) u(t) dt + \lambda_{2} \int_{a}^{b} K_{2}(x,t) u(t) dt\label{eq:2_1}
	\end{eqnarray}
	and
	\begin{eqnarray}
		u(x,t) = f(x,t) + \lambda \int_{0}^{t} \int_{\Omega}F(x,t,\epsilon ,\tau , u(\epsilon ,\tau)) d\epsilon d\tau, (x,t) \in \Omega \times [0,T]\label{eq:2_2}
	\end{eqnarray}
	where $f(x,t)$ and $F(x,t,\epsilon , \tau,u(\epsilon , \tau))$ are analytic functions on $D = \Omega \times [0,T]$, and $\Omega$ is a closed subset of $R^n, n =1 ,2,3$. It is interesting to note that equation \refx{2_1} contains disjoint Volterra and Fredholm integral equations, whereas equation \refx{2_2} contains mixed Volterra and Fredholm integral equations. Moreover, the unknown functions $u(x)$ and $u(x,t)$ appear inside and outside the integral signs.\\
	
	\NI This is a characteristic feature of a second kind integral equation. If the unknown functions appear only inside the integral signs, the resulting equations are of first kind, but will not be examined in this text. Examples of the two types are given by
	\begin{eqnarray}
		u(x) = 6x + 3x^2 + 2 - \int_{0}^{x} xu(t)dt - \int_{0}^{1} tu(t)dt\label{eq:2_3}
	\end{eqnarray}
	and
	\begin{eqnarray}
		u(x,t) = x + t^3 + \frac{1}{2} t^2 - \frac{1}{2} t - \int_{0}^{t} \int_{0}^{1} (\tau - \epsilon)dEdT\label{eq:2_4}
	\end{eqnarray}
	
	\section{Singular Integral Equations}
	Volterra integral equations of the first kind 
	%$\) \textbf{(2.4.1)}\\
	\begin{eqnarray}
		f(x) = \lambda \int_{g(x)}^{h(x)} K(x,t)u(t)dt\label{eq:2_5}
	\end{eqnarray}
	or of the second kind
	%$\) \textbf{(2.4.2)}\\
	\begin{eqnarray}
		u(x) = f(x) + \int_{g(x)}^{h(x)} K(x,t)u(t)dt\label{eq:2_6}
	\end{eqnarray}
	are called singular if one of the limits of integration $g(x), h(x)$ or both are infinite. Moreover, the previous two equations are called singular if the kernel $K(x,t)$ becomes unbounded at one or more points in the interval of integration. In this text we will focus our concern on equations of the form:
	%\textbf{(2.4.3)}\\
	\begin{eqnarray}
		f(x) =\int_{0}^{x} \frac{1}{(x - t)^\alpha} u(t)dt ~~, ~~~~~~0 < \alpha < 1
	\end{eqnarray}
	or of the second kind:
	%\textbf{(2.4.4)}\\
	\begin{eqnarray}
		u(x) = f(x) +\int_{0}^{x} \frac{1}{(x - t)^ \alpha} u(t)dt ~~~~,   ~~~~~~~~ 0 < \alpha < 1
	\end{eqnarray}
	
	\NI The last two standard forms are called generalized Abel’s integral equation and weakly singular integral equations respectively. For $\alpha = \frac{1}{2}$ , the equation:
	%\textbf{(2.4.5)}\\
	\begin{eqnarray}
		f(x) =\int_{0}^{x} \frac{1}{\sqrt{x - t}} u(t)dt
	\end{eqnarray}
	is called the Abel’s singular integral equation. It is to be noted that the kernelin each equation become infinity at the upper limit $t = x$. Examples of Abel’s integral equation, generalized Abel’s integral equation, and the weakly singular integral equation are given by
	%\textbf{(2.4.6)}   \textbf{(2.4.7)}\\\\
	\begin{gather}
		\sqrt{x} =\int_{0}^{x} \frac{1}{\sqrt{x - t}} u(t)dt\sps
		x^3 = \int_{0}^{x} \frac{1}{(x - t)^\frac{1}{3}} u(t)dt
	\end{gather}		
	and
	%\textbf{(2.4.8)}\\
	\begin{eqnarray}
		u(x) = 1 + \sqrt{x} +\int_{0}^{x} \frac{1}{(x - t)^\frac{1}{3}} u(t)dt
	\end{eqnarray}
	respectively.

	\section{Method of Solving Volterra Integral Equations}
	\subsection{The Adomian Decomposition Method}
	The Adomian decomposition method (ADM) was introduced and developed by George Adomian in 1970s to 1990s and is well addressed in many references. A considerable amount of research work has been invested recently in applying this method to a wide class of linear and non-linear ordinary differential equations, partial differential equations and integral equations as well. The Adomian decomposition method consists of decomposing the unknown function $u(x)$ of any equation into a sum of an infinite number of components defined by the decomposition series
	%$$			. . . . . . 1\\
	\begin{equation}
		u(x) = \sum_{n=0}^{\infty}  u_n(x)\label{eq:2_13}
	\end{equation}
	or equivalently
	%$$ . . . ,		. . . . . . .2\\
	\begin{eqnarray}
		u(x) = u_0(x) + u_1(x) + u_2(x) + \cdots \label{eq:2_14}
	\end{eqnarray}
	where the components $u_n(x)$, $n\geq 0$ are to be determined  in a recursive manner. The decomposition method concerns itself with finding the components $u_0, u_1, u_2,\ldots$,  individually.As will be seen through the text, the determination of these components can be achieved in an easy way through a recurrence relation that usually involves simple integrals that can be easily evaluated.\\
	
	\NI To establish the recurrence relation, we substitute \refx{2_13} into the Volterra integral equation of the second kind to obtain
	%\(,	. . . .3\\
	\begin{eqnarray}
		\sum_{n=0}^{\infty} U_{n}(x)= f(x) + \lambda \int_{0}^{x} K(x,t) \sum_{n=0}^{\infty} U_{n}(t)dt
	\end{eqnarray}
	or equivalently
	%$\).		. . .  . .4\\
	\begin{multline}
		u_0(x) + u_1(x) + u_2(x) + \cdots = f(x) + \lambda \int_{0}^{x} K(x,t) [u_0(t) \\+ u_1(t) + \cdots]dt
	\end{multline}
	The zeroth component $u_0(x)$ is identified by all terms that are not included under the integral sign. Consequently, the components $u_j(x), j \geq 1$ of the unknown function $u(x)$ are completely determined by setting the recurrence relation:
	%	. . . . . .5\\
	\begin{eqnarray}
		\begin{split}
				u_o(x) &= f(x)\sps
			u_{n+1}(x) &= \lambda\int_{0}^{x} K(x,t) u_n(t) dt,~~~ n \geq 0
		\end{split}
	\end{eqnarray}
	that is equivalent to
	% . . . . . . 6\\
	\begin{eqnarray}
		\begin{split}
			u_0(x) &= f(x)\sps
			u_1(x) &=  \lambda\int_{0}^{x} K(x,t) u_0(t) dt\sps
			u_2(x) &=  \lambda\int_{0}^{x} K(x,t) u_1(t)dt\sps
			u_3(x) &=  \lambda\int_{0}^{x} K(x,t) u_2(t) dt
		\end{split}\label{eq:2_18}
	\end{eqnarray}
	and so on for the other components.\\
	
	\NI In view of equations \refx{2_18}, the components $u_0(x),u_1(x),u_2(x),u_3(x),\ldots$ are completely determined. As a result, the solution $u(x)$ of the Volterra integral equation of the second type in a series form is readily obtained by using the series assumption in equation \refx{2_13}. It is clearly seen that the decomposition method converted the integral equation into an elegant determination of computable components. It was formally shown by many researchers that if an exact solution exists for the problem,then the obtained series converges very rapidly to that solution.The convergence concept of the decomposition series was thoroughly investigated by many researchers to confirm the rapid convergence of the resulting series.\\ 
	\NI However, for concrete problems, where a closed form solution is not obtainable, a truncated number of terms is usually used for numerical purposes. The more components we use the higher accuracy we obtain. \\
	
	\subsection{The Modified Adomian Decomposition Method}
	As shown before, the Adomian decomposition method provides the solution in an infinite series of components. The components $u_j,j \geq 0$ are easily computed if the inhomogeneous term $f(x)$ in the Volterra integral equation: 
	%\)			. . . . . . . 7\\
	\begin{eqnarray}
		u(x) = f(x) +  \lambda\int_{0}^{x} K(x,t) u(t) dt
	\end{eqnarray}
	consists of a polynomial. However, if the function $f(x)$ consists of a combination of two or more of polynomials, trigonometric functions, hyperbolic functions, and others, the evaluation of the components $u_j,j \geq 0$ requires cumbersome work. A reliable modification of the Adomian decomposition method was developed by Wazwaz. The modified decomposition method will facilitate the computational process and further accelerate the convergence of the series solution. The modified decomposition method will be applied, wherever it is appropriate, to all integral equations and differential equations of any order. It is interesting to note that the modified decomposition method depends mainly on splitting the function $f(x)$ into two parts, therefore it cannot be used if the function $f(x)$ consists of only one term.\\
	
	\NI To give a clear description of the technique, we recall that the standard Adomian decomposition method admits the use of the recurrence relation:
	%		. . . . . .8\\
	\begin{eqnarray}
		\begin{split}
			u_o(x) &= f(x)\sps
			u_{k+1}(x) &= \lambda\int_{0}^{x} K(x,t) u_k(t) dt\sps
			&k \geq 0
		\end{split}\label{eq:2_20}
	\end{eqnarray}
	where the solution $u(x)$ is expressed by an infinite sum of components defined before by;\\
	%$$			. . . . . . 9\\
	\begin{eqnarray}
		u(x) = \sum_{n=0}^{\infty}  u_n(x)
	\end{eqnarray}
	In view of \refx{2_20}, the components $u_n(x),n\geq 0$ can be easily evaluated. The modified decomposition method introduces a slight variation to the recurrence relation \refx{2_20} that will lead to the determination of the components of $u(x)$ in an easier and faster manner. For many cases, the function $f(x)$ can be set as the sum of two partial functions, namely $f_1(x)$ and $f_2(x)$. In other words, we can set
	%	. . . . . .  .10\\
	\begin{eqnarray}
		f(x) = f_1(x) + f_2(x)\label{eq:2_22}
	\end{eqnarray}
	In view of \refx{2_22}, we introduce a qualitative change in the formation of the recurrence relation \refx{2_20}. To minimize the size of calculations, we identify the zeroth component $u_0(x)$ by one part of $f(x)$, namely $f_1(x)$ or $f_2(x)$. The other part of f(x) can be added to the component $u_1(x)$ among other terms. In other words, the modified decomposition method introduces the modified recurrence relation: 
	%. . . . . . .11\\
	\begin{eqnarray}
		\begin{split}
			u_0(x) &= f_1(x)\sps
			u_1(x) &= f_2(x) + \lambda\int_{0}^{x} K(x,t) u_0(t) dt\sps
			u_k+1(x) &= \lambda\int_{0}^{x} K(x,t) u_k(t) dt\sps
			&k \geq 1
		\end{split}\label{eq:2_23}
	\end{eqnarray}
	
	\NI This shows that the difference between the standard recurrence relation \refx{2_20} and the modified recurrence relation \refx{2_23} rests only in the formation of the first two components $u_0(x) and u_1(x)$ only. The other components $u_j, j \geq 2$ remain the same in the two recurrence relations. Although this variation in the formation of $u_0(x) and u_1(x)$ is slight, however it plays a major role in accelerating the convergence of the solution and in minimizing the size of computational work. Moreover,reducing the number of terms in $f_1(x)$ affects not only the component $u_1(x)$, but also the other components as well. This result was confirmed by several research works. Two important remarks related to the modified method can be made here. First, by proper selection of the functions $f_1(x)$ and $f_2(x)$, the exact solution $u(x)$ may be obtained by using very few iterations, and sometimes by evaluating only two components. The success of this modification depends only on the proper choice of $f_1(x)$ and $f_2(x)$, and this can be made through trials only. A rule that may help for the proper choice of $f_1(x)$ and $f_2(x)$ could not be found yet. Second, if $f(x)$ consists of one term only, the standard decomposition method can be used in this case.
	It is worth mentioning that the modified decomposition method will be used for Volterra and Fredholm integral equations, linear and non-linear equations.\\
	
	\subsection{The Noise Terms Phenomenon:}
	It was shown before that the modified decomposition method presents a reliable tool for accelerating the computational work.However, a proper selection of $f_1(x)$ and $f_2(x)$ is essential for a successful use of this technique. A useful tool that will accelerate the convergence of the Adomian decomposition method is developed. The new technique depends mainly on the so-called noise terms phenomenon that demonstrates a fast convergence of the solution. The noise terms phenomenon can be used for all differential and integral equations. The noise terms, if existed between the components $u_0(x)$ and $u_1(x)$, will provide the exact solution by using only the first two iterations. In what follows, we outline the main concepts of the noise terms:
	\begin{enumerate}
		\item The noise terms are defined as the identical terms with opposite signs that arise in the components $u_0(x) and u_1(x)$. Other noise terms may appear between other components. As stated above, these identical terms with opposite signs may exist for some equations, and may not appear for other equations.
		
		\item  By cancelling the noise terms between $u_0(x)$ and $u_1(x)$, even though $u_1(x)$ contains further terms, the remaining non-cancelled terms of $u_0(x)$ may give the exact solution of the integral equation. The appearance of the noise terms between $u_0(x) and u_1(x)$ is not always sufficient to obtain the exact solution by cancelling these noise terms. Therefore, it is necessary to show that the non-cancelled terms of $u_0(x)$ satisfy the given integral equation. On the other hand, if the non-cancelled terms of $u_0(x)$ did not satisfy the given integral equation, or the noise terms did not appear between $u_0(x)$ and $u_1(x)$, then it is necessary to determine more components of $u(x)$ to determine the solution in a series form as presented before.
		
		\item It was formally shown that the noise terms appear for specific cases of inhomogeneous differential and integral equations, whereas homogeneous equations do not give rise to noise terms. The conclusion about the self-cancelling noise terms was based on solving several specific differential and integral models. However, a proof for this conclusion was not given.
	\end{enumerate}
	
	\NI A useful summary about the noise terms phenomenon can be drawn as follows:
	\begin{enumerate}
		\item The noise terms are defined as the identical terms with opposite signs that may appear in the components $u_0(x)$ and $u_1(x)$ and in the other components as well.
		
		\item The noise terms appear only for specific types of inhomogeneous equations whereas noise terms do not appear for homogeneous equations.
		
		\item Noise terms may appear if the exact solution of the equation is part of the zeroth component $u_0(x)$.
		
		\item \textbf{The Laplace Transform Method:}
		The Laplace transform method is a powerful technique that can be used for solving initial value problems and integral equations as well. In the convolution theorem for the Laplace transform, it was stated that if the kernel $K(x,t)$ of the integral equation: 
		\begin{eqnarray*}
			u(x) = f(x) + \lambda\int_{0}^{x} K(x,t) u(t) dt
		\end{eqnarray*}
		depends on the difference $x-t$, then it is called a difference kernel. Examples of the difference kernel are $e^{x-t},\cos(x - t), \sin(x - t)$. The integral equation can thus be expressed as
		\begin{eqnarray*}
			u(x) = f(x) + \lambda\int_{0}^{x} K(x - t) u(t) dt
		\end{eqnarray*}
		
		\NI Consider two functions $f_1(x)$ and $f_2(x)$ that possess the conditions needed for the existence of Laplace transform for each. Let the Laplace transforms for the functions $f_1(x)$ and $f_2(x)$ be given by: \\
		\begin{eqnarray*}
			\begin{split}
				\Laplace{f_1(x)} &= F_1(s),\sps
				\Laplace{f_2(x)} &= F_2(s)\sps
			\end{split}
		\end{eqnarray*}	
		The Laplace convolution product of these two functions is defined by
		\begin{eqnarray*}
			(f_1 * f_2)(x) =\int_{0}^{x} f_1(x - t) f_2(t) dt
		\end{eqnarray*}
		or
		\begin{eqnarray*}
			(f_2 * f_1)(x) =\int_{0}^{x} f_2(x - t) f_1(t) dt
		\end{eqnarray*}
		Recall that;
		\begin{eqnarray*}
			(f_1 * f_2)(x) = (f_2 * f_1)(x)
		\end{eqnarray*}
		We can easily show that the Laplace transform of the convolution product $(f1 * f2)(x)$ is given by
		\begin{eqnarray*}
			\Laplace{(f_1 * f_2)(x)} = \Laplace\left\{\int_{0}^{x} f_1(x - t) f_2(t) dt\right\} = F_1(s)F_2(s)
		\end{eqnarray*}
	\end{enumerate}
	


	%%%%%%%%%%%%%%%%%%%CHAPTER THREE%%%%%%%%%%%%%%%%%%%
	\chapter{METHODOLOGY}
	\section{Laplace Transform}
	In mathematics, the Laplace transform, named after its inventor Pierre-Simon Laplace, is an integral transform that converts a function of a real variable $t$ (often time) to a function of a complex variable $s$ (complex frequency). The transform has many applications in science and engineering because it is a tool for solving differential equations. In particular, it transforms linear differential equations into algebraic equations and convolution into multiplication.

	
	%%%%%%%%%%%%%%%%%%%CHAPTER FOUR%%%%%%%%%%%%%%%%%%%
	\chapter{APPLICATION OF DIFFERENTIAL EQUATIONS TO ELECTRICAL AND MECHANICAL ENGINEERING}
	\section{INTRODUCTION}
	After a period of intense internal development which lead to an unpreceded depending of mathematics, the last few decades 

	%%%%%%%%%%%%%%%%%%%CHAPTER FIVE%%%%%%%%%%%%%%%%%%%
	\chapter{SUMMARY, CONCLUSION AND RECOMMENDATION}
	\section{SUMMARY}
	In this project, chapter one provided a general introduction to differential equations with related motivations and concepts.\\
	
	\NI Chapter two was used to elaborate on types, methods, and examples of first order differential equation and\\
	
	\NI A brief account of some first application of differential equation to biological, mechanical and electrical engineering were presented and solved in chapter three and four.
	
	
	\section{CONCLUSION}
	The most important branch of mathematics used for mathematical formulation is the differential equation.  Any physical situation involved motion or measure rates of change can be described by a mathematical model, the model is just a differential equation.This equation effectively related the quality or function upon which the attention is focused with the independent variable such as time, position upon which it may depend. Thus, the study of ordinary differential equation cannot be ignored and in this project, we were able to solved some biological and engineering problems using ordinary differential equation.\\
	
	
	\section{RECOMMENDATION}
	The application of differential equation in biological, mechanical and electrical engineering is recommended for organizations such as ministry of health, work and non-governmental organization. The area is fertile in terms of research and we therefore recommend students and researchers to venture into this area. Consequently, the government of Nigeria should also look into this area and motivate people in it.  
	
	
	%%%%%%%%%%%%%%%%%%%REFERENCE%%%%%%%%%%%%%%%%%%%
	\chapter*{REFERENCES}
	\addcontentsline{toc}{chapter}{REFERENCES}
	
	\begin{description}
		\item Dass, H. K. (1988). \emph{Advanced Engineering Mathematics} (1st ed., pp. 154–230). S. CHAND \& COMPANY PVT. LTD.
		
		\item Frank Ayres (1952). Schum's Outline Series, Theory and Problems of Differential Equations.
		
		\item Frank Hoppenstead (1995). A Journal Getting Started to Mathematical Biology.
		
		\item Gentry R.D (1978). \emph{Introduction to Calculus for The Biological and Health Sciences}, Addition (Wesley Publishing Company).
		
		\item G. William. An Introduction to Electrical Circuit Theory.
		
		\item Ince E.L (1956). \emph{Ordinary Differential Equations 4th Edition}.
		
		\item Leigh E.R (1968). The Ecological Role of Volterra’s Equation In Lectures On Mathematics In The Life Science Some Mathematical Problems In Biology.
		
		\item Lotka, A. J. (1957). \emph{Elements of Mathematical Biology}.
		
		\item Riley, K. F. (2012). Mathematical Methods for the Physical Sciences. \emph{An Informal Treatment for Students of Physics and Engineering}. https://doi.org/10.1017/CBO9781139167550
		
		\item Ritger, P. D., \& Rose, N. J. (2010). \emph{Differential Equations with Applications}.
		
		\item Rainville, E., Bedient, P., \& Bedient, R. (1996). \emph{Elementary Differential Equations}. Pearson. https://doi.org/10.1604/9780135080115.
		
		\item Stroud, K., \& Booth, D. (2001). Engineering Mathematics. \emph{Programmes and Problems}.
		
		\item Stroud, K. A., \& Booth, D. J. (2013). \emph{Engineering Mathematics}. https://doi.org/10.1057/978113703122810.1057/978-1-137-03122-8
		
		\item Thomas A.B \& Finney, R. (1988). Calculus and Analytic Geometry 7th Editions. Reading Addison Wesley.
		
		

		
	\end{description}
	
\end{document}