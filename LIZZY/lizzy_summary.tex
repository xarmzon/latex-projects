\documentclass[12pt]{report}
\usepackage{amsmath}
\usepackage{amssymb}
\usepackage{graphicx}
\usepackage{longtable}
\usepackage{tikz}

\newcommand{\bt}[1]{\textbf{#1}}
\newcommand{\ubt}[1]{\textbf{\underline{#1}}}
\newcommand{\sps}{\\[0.2cm]}
\newcommand{\spn}[1]{\\[#1cm]}
\newcommand{\refn}[1]{(\ref{#1})}
\newcommand{\refx}[1]{\refn{eq:#1}}
\newcommand{\NI}{\noindent}
\newcommand{\dsp}{\displaystyle}
\newcommand{\sprime}{'}
\newcommand{\dprime}{''}
\newcommand{\tprime}{'''}
\newcommand{\tti}[1]{\textit{#1}}


\renewcommand*\contentsname{Table of Contents}
\renewcommand{\baselinestretch}{1.4}

\begin{document}
	%%remove the numbering from the first page 
	\clearpage
	\thispagestyle{empty}
	%%TITLE%%
	\addcontentsline{toc}{chapter}{TITLE PAGE}
	\begin{center}
		\textbf{\itshape SUMMARY ON}
	\end{center} 
	\begin{center}
		{\bf \Large EFFECTS OF NON-NEWTONIAN PARAMETER ON UNSTEADY MHN OSCILLATORY OF VISCOELASTIC FLUIDS IN AN INCLINED PLANER CHANNEL}
	\end{center}
	$$$$
	\vspace{3cm}
	\begin{center}
		\textbf{\itshape BY}
	\end{center} 
	$$$$
	\vspace{2cm}
	\begin{center}
		{\bf KODJO, Elizabeth\\
			17/56EB062}
	\end{center}
	$$$$
	\\ \\
	%\newpage
	%\pagenumbering{roman} 
	%%TABLE OF CONTENTS%%
	%\addcontentsline{toc}{chapter}{TABLE OF CONTENTS}
	%\tableofcontents
	\newpage
	
	\pagenumbering{arabic}
	%%%%%%%%%%%%%%%%%%%%%%%%%CHAPTER ONE%%%%%%%%%%%%%%%%%%%%%%%%%%%%
	\chapter{FLUID MECHANICS}
	Fluid mechanics is a branch of physics that studies the mechanics of fluids and the forces on them. It is divided into two; fluid statics and fluid dynamics.\sps
	Fluid mechanics can be mathematically complex and can be best solved using numerical methods.
	
	\section{History of Fluid Mechanics}
	The study of fluid started in the days of ancient Greece when Archimedes (285 - 212 BC) investigated fluid statics and formulated his famous law; \bt{ARCHIMEDES PRINCIPLE}. Floating bodies was the first major on fluid mechanics. In 1620 - 1684, a French man built the first wind tunnel and tested model on it.
	
	\section{Main Branches of Fluid Mechanics}
	\begin{enumerate}
		\item [i] \bt{Fluid statics:} is a branche of fluid mechanics that deals with the study of fluid at rest. i.e where fluid is not in motion and it also gives reason why pressure change, oil on water in our everyday life.
		
		\item [ii] \bt{Fluid dynamics:} is a branch of fluid mechanics that deals with the study of fluid in motions. It application includes the calculating of forces and moments on aircraft, the mass flow rate of petroleum through pipelines and many more.
	\end{enumerate}

	\section{Fluids}
	Fluid is a substance that continually flow under an applied force and when the flow into a solid body they take the shape of the solid body because of their inability to support a SDFSD in state of balance.
	
	\subsection{Properties of Fluids}
	(i) Temperature (ii) Pressure (iii) Density (iv) Viscosity (v) Specific Volume

	\subsection{Types of Fluids}
	\begin{enumerate}
		\item \bt{Ideal Fluid:} This is a fluid that has no viscosity
		\item \bt{Real Fluid:} This is a fluid that has viscosity e.g kerosene, Petrol and CASDSD oil
		\item Newtonian fluid
		\item Non-Newtonian fluid
	\end{enumerate}

	\subsection{Fluid flow}
	Fluid flow has all kinds of aspects steady or unsteady, compression \& incompressible.
	
	\section{Objective of Study}
	This study is to establish the effect of Non-Newtonian Parameter on unsteady MHD Oscillatory slip flow of Viscoelastic fluids in an inclined planer channel filled with salor
	
	\section{Introduction}
	The study of oscillatory fluid flow in a inclined channel filled with porous medium has been given more attention for the past year and other fields $\dsp V_x = (\frac{dV_x}{dy})$ is limits\sps
	So here we need to investigate the effect of Non-Newtonian parameter on MHD.

	\section{Definition of Terms}
	\bt{MAGNET HYDRODYNAMICS(MHD)} is the study of the magnetic properties of electrical conducing fluids.\sps
	\bt{Slip Flow:} this indicates that the fluid particle do not stick to the surface during flow\sps
	\bt{Viscoelatic Fluid:} It is a type of fluid particle which behaves in a manner to Newtonian.


	%%%%%%%%%%%%%%%%%%%%%%%%%CHAPTER TWO%%%%%%%%%%%%%%%%%%%%%%%%%%%%
	\chapter{LITERATURE REVIEW}
	The flow of viscoelastic fluid through porous medium has attracted the attention of many people. So the purpose of this work is to investigate the effect of Non-Newtonian parameter on unsteady MHD oscillatory slip flow of viscoelastic fluid in an inclined channel.
	
	\section{Governing Equations}
	We have three equations which governs fluid dynamics
	
	\section{Continuity Equation}
	\section{Momentum Equations}
	\section{Energy equation}
	\begin{equation}
		= \frac{\partial \vec{V}}{\partial T} + (\vec{V} \cdot \nabla)\vec{V} = \frac{-1}{P}\nabla P + U\nabla^2 + F
	\end{equation}
	
	\section{Maxwell's Equation}
	They describe how electric and magnetic field are generated and altered by each other current \& charges.
	
	\section{Characteristic Number}
	\begin{enumerate}
		\item REYNOIDS Numbers: $\dsp R_e = \rho \frac{uL}{\mu}$ It is used to measure how turbulent flow is
		
		\item Mach Number: $\dsp M = \frac{u}{c}$ It measure the flow of compressibility
		
		\item Peclet Number
		
		\item Schmidt Number
	\end{enumerate}


	%%%%%%%%%%%%%%%%%%%%%%%%%CHAPTER THREE%%%%%%%%%%%%%%%%%%%%%%%%%%%%
	%\chapter{MATHEMATICAL FORMULATION}

	%%%%%%%%%%%%%%%%%%%%%%%%%CHAPTER FOUR%%%%%%%%%%%%%%%%%%%%%%%%%%%%
	%\chapter{}

\end{document}

