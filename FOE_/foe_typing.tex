\documentclass[a4paper 11pt]{article}

%packages needed for the math and math symbols
\usepackage{amsmath}
\usepackage{amssymb}

\newcommand{\Laplace}{\mathcal{L}}
\newcommand{\ft}{f(t)}
\newcommand{\ftn}[1]{f^{#1}(t)}
\newcommand{\FS}{F(S)}
\newcommand{\FSn}[1]{F^{#1}(S)}
\newcommand{\LaplaceIntegral}{\int_{0}^{-\infty}e^{-st}\ft\text{dt}}
\newcommand{\sbracket}[1]{\left[#1\right]}
\newcommand{\Y}{Y}
\newcommand{\Yn}[1]{Y^{#1}(0)}
\newcommand{\fn}[1]{f^{#1}(0)}
\newcommand{\Sn}[1]{S^{#1}}
\newcommand{\LUx}[1]{\Laplace\sbracket{U_{#1}(x,t)}}
\newcommand{\Un}[2]{U_{#1}(#2)}
\newcommand{\LInt}{\int_{0}^{\infty}e^{-st}U(x,t)\text{dt}}
\newcommand{\LFt}{\Laplace \sbracket{\ftn{}}}
\newcommand{\NI}{\noindent}
\newcommand{\LFn}[1]{\Laplace \sbracket{#1}}
\newcommand{\Dtl}[1]{\Delta_{#1}}
\newcommand{\sqL}[1]{#1^{2}}
\newcommand{\sqLn}[2]{#1^{#2}}
\newcommand{\psq}{\pi^{2}}
\newcommand{\psqn}[1]{\pi^{#1}}
\newcommand{\InverseL}[1]{\Laplace^{-1}\left[#1\right]}
\newcommand{\LT}[1]{\Laplace \left[#1\right]}
\newcommand{\LTb}[1]{\Laplace \left(#1\right)}
\begin{document}

\begin{center}
\textbf{APPLICATION OF LAPLACE TRANSFORM METHOD IN SOLVING SECOND ORDER PARTIAL DIFFERENTIAL EQUATION} \\[1.5cm]
\end{center}

\NI \underline{\textbf{Laplace Method:}}\\[0.2cm]

\NI $\Laplace\left[\ft\right] = \FS = \LaplaceIntegral$ \\[0.1cm]
$ \Laplace \sbracket{\ftn{n}}  = S^{n}Y - \Sn{n-1}Y(0) - \Sn{n-2}\Yn{\prime} \cdots\cdots - S\fn{n-2} \cdots\cdots - y^{n-1}(0) $ \\[0.1cm]
$\LUx{x} = \LInt \equiv U(x,s) $\\[0.1cm]
$\LUx{x} = \Un{x}{x,s}$\\[0.1cm]
$\LUx{x} = \Un{x}{x,s}$\\[0.1cm]
$\LUx{t} = S\Un{}{x,s} - \Un{}{x,0}$\\[0.1cm]
$\LUx{t} = S^{2}\Un{}{x,s} - S\Un{}{x,0} - \Un{t}{x,0}$\\[0.1cm]
$F(s) = \LFt $ then,\\[0.2cm]
$\Laplace\sbracket{\Un{}{t-a}\centerdot g(t-a)} = e^{-as}G(s)$\\[.6cm]

\NI \underline{\textbf{P.D.E of Order 2}}\\[0.1cm]

\NI \underline{\textbf{Examples:}}\\[0.2cm]

\NI (1) ${\displaystyle \frac{\partial^{2}u}{\partial x^{2}}(x,t) = \frac{\partial u}{\partial x}(x,t), 0 < x < 2, t > 0  \text{ \text{ }} U(0,t) = 0, U(2,t) = 0, U(x,0) = 3\sin(2\pi x)}$\\

\NI \underline{\textbf{Solution:}}

\NI $\Un{xx}{x,t} = \Un{t}{x,t}$\\[0.2cm]
taking the Laplace transform \\[0.2cm]
$\LFn{\Un{xx}{x,t}} = \LFn{\Un{t}{x,t}}$\\[0.2cm]
$\Un{xx}{x,s} = s\Un{}{x,s} - \Un{}{x,0}$\\[0.7cm]
Using the condition, $\Un{}{x,0} = 3\sin(2\pi x)$ \\[0.2cm]
$S\Un{}{x,s} - 3\sin(2\pi x) = \Un{xx}{x,s}$ \\[0.2cm]
$\Longrightarrow \Un{xx}{x,s} - s\Un{}{x,s} = - 3\sin(2\pi x)$ \\[0.2cm]
${\displaystyle \frac{d^{2}u}{dx^{2}} - sU = - 3\sin(2\pi x)}$\\[0.7cm]
Solving the Homogenous Problem \\[0.2cm]
${\displaystyle \frac{d^{2}u}{dx^{2}} - sU = 0}$\\[0.7cm]
The characteristic equation is given by \\[0.2cm]
$m^2 - s = 0 \Rightarrow m = \pm \sqrt{s}$ \\[0.2cm]

\NI The homogenous solution is: \\[0.2cm]
$$\Un{A}{x,s} = A_{1}e^{\sqrt{sx}} + A_{2}e^{-\sqrt{sx}}$$

\NI Solving the non-homogenous problem using the method of Undetermined Coefficient \\[0.2cm]
i.e ${\displaystyle \frac{d^{2}u}{dx^{2}} - sU = - 3\sin(2\pi x) - - - - - - - - - - - - (*)}$ \\[0.2cm]
Let \\ ${\displaystyle U = \Delta_{1}\sin(2\pi x) + \Delta_{2}\cos(2\pi x) - - - - - - - - - (a)}$ \\[0.2cm]
 ${\displaystyle U^{\prime} = 2\pi \Delta_{1}\cos(2\pi x) - 2\pi \Delta_{2}\sin(2\pi x)  - - - - - - - (b)}$ \\[0.2cm]
 ${\displaystyle U^{\prime\prime} = -4\pi^{2} \Delta_{1}\sin(2\pi x) - 4\pi^2\Delta_2\cos(2\pi x)  - - - - - - (c)}$ \\[0.2cm]

\NI Substituting (a) and (c) in equation (*) \\[0.2cm]
$-4\pi^{2}\Dtl{1}\sin(2\pi x) - 4\pi^{2}\Dtl{2}\cos(2\pi x) - S\Dtl{1}\sin(2\pi x) - S\Dtl{2}\cos(2\pi x) = -3\sin(2\pi x)$ \\[0.2cm]
$-4\pi^{2}\Dtl{1} - S\Dtl{1} = - 3$  \quad \quad \quad Also, $4\pi^{2}\Dtl{2} - S\Dtl{2} = 0$ \\[0.2cm]
$-\Dtl{1}\sbracket{4\pi^{2} + s} = - 3 $ \quad \quad \quad \quad \quad \quad $\Dtl{2}\sbracket{s + 4\pi^{2}} = 0$ \\[0.2cm]
$\Dtl{1} = \displaystyle \frac{3}{s + 4\pi^{2}} $ \quad \quad \quad \quad \quad \quad \quad  \quad \quad $\Dtl{2} = 0$ \\[0.2cm]

\NI The particular solution is:\\
$$\Un{p}{x,s} = \frac{3}{s + 4\pi^{2}}\sin(2\pi x) $$

\NI  \textbf{The general solution is given by}:

$$\Un{}{x,s} = A_{1}e^{\sqrt{sx}} + A_{2}e^{-\sqrt{sx}} + \frac{3\sin(2\pi x)}{s + 4\pi^{2}}$$ \\[0.5cm]

\NI Applying the boundary conditions $\Un{}{0,t} = 0, \; \Un{}{2,t} = 0$

\NI $\displaystyle \Un{}{0,s} = A_1 + A_2 = 0 \Rightarrow A_1 = - A_2$\\[0.2cm]
\NI $\displaystyle \Un{}{2,s} = A_{1}e^{2\sqrt{s}} + A_{2}e^{-2\sqrt{s}} = 0 \; \; \; \sbracket{\text{But } A_1 = - A_2}$\\[0.2cm]
$\displaystyle - A_2e^{2\sqrt{s}} + A_2e^{-2\sqrt{s}} = 0$\\[0.2cm]
$\displaystyle A_{2}\sbracket{e^{-2\sqrt{s}} - e^{2\sqrt{s}}} = 0 \; \; \Rightarrow A_2 = 0, A_1 = 0$\\[0.2cm]

\NI ${\displaystyle \Un{}{x,s} = \frac{3\sin(2\pi x)}{s + 4\pi^{2}}}$\\[0.5cm]

\NI Subtituing (a) and (c) in equation (*)\\
${\displaystyle -c^{2}\psq n_{1}\sin(\pi x) - c^{2}\psq n_{2}\cos(\pi x) - s^{2}n_{1}\sin(\pi x) -  s^{2}n_{2}\cos(\pi x) = \frac{-\sin(\pi x)}{s}}$\\[0.2cm]
$ \Longrightarrow \displaystyle -c^{2}\psq n_{1} - s_{2}n_{1}  = \frac{-1}{s} \Rightarrow -n_{1}\left[s^{2} + c^{2}\psq \right] = \frac{-1}{s} \\ 
\Longrightarrow n_{1} = \frac{1}{s\left[s^{2} + c^{2}\psq \right]}
$\\[0.2cm]

\NI Also, 
$
\displaystyle -c^{2}\psq n_{2}\cos(\pi x) - s^{2}n_{2}\cos(\pi x) = 0 \\[0.2cm]
n_2\left[s^{2} + c^{2}\psq \right] = 0 \\[0.2cm] 
\Longrightarrow n_{2} = 0 \\
$\\[0.2cm]

\NI Substituting '$n_{1}$' and '$n_{2}$' in (a)\\[0.2cm]
$ \displaystyle
\Un{p}{x,s} = \frac{\sin(\pi x)}{s}
$\\[0.2cm]

\NI \textbf{The general solution is given as:}\\[0.2cm]
$\displaystyle
\Un{g}{x,s} = \Un{u}{x,s} + \Un{p}{x,s} \\[0.2cm]
\Un{g}{x,s} = A_{1}e^{\frac{sx}{c}} + A_{2}e^{-\frac{sx}{c}} + \frac{\sin(\pi x)}{s\left[ s^{2} + c^{2}\psq \right]} \; - - - - - - - - - - - - - - (**)
$\\[0.2cm]

\NI Apply the boundary conditions;
$\displaystyle
\; \; \Un{}{0,t} = 0 \; \text{and} \; \Un{}{1,t} = 0 \\[0.2cm]
\Un{}{0,s} = A_1 + A_2 = 0 \; \; \Longrightarrow A_1 = - A_2 \\[0.1cm]
\Un{}{1,s} = A_1 e^{\frac{s}{c}} + A_2 e^{- \frac{s}{c}} = 0 \Rightarrow A_2 = 0 \Rightarrow A_1 = 0
$\\[0.3cm]

\NI Substituting $'A_1'$ and $'A_2'$ in eqution (**) \\[0.2cm]
$\displaystyle
\Un{}{x,s} = \frac{\sin(\pi x)}{s \left[s^{2} + c^{2}\psq \right]}
$\\[1cm]

\NI Applying Inverse Laplace Transform \\[0.2cm]
$\displaystyle
\InverseL{\Un{}{x,s}} = \sin(\pi x) \; \InverseL{\frac{1}{s\sbracket{s^{2} + c^{2}\psq}}}
$\\[0.3cm]

\NI Resolving $\displaystyle \frac{1}{s\sbracket{s^{2} + c^{2}\psq}}$ into partial fractions \\[0.2cm]
$\displaystyle
\frac{1}{s\sbracket{s^{2} + c^{2}\psq}} = \frac{A}{s} + \frac{Bs + D}{s^{2} + c^{2}\psq} = \frac{A\sbracket{s^2 + c^2 \psq} + \sbracket{Bs + D}s}{s\sbracket{s^2 + c^2 \psq}}
$\\[0.3cm]
$\displaystyle
1 = A \sbracket{s^2 + c^2 \psq} + Bs^2 + Ds
$\\[0.3cm]

\NI Taking the Inverse Laplace equation\\[0.2cm]
$\displaystyle
\InverseL{\Un{}{x,s}} = \InverseL{\frac{3\sin(2\pi x)}{s + 4\psq}}
$\\[0.3cm]

$$ \mathbf{
\Un{}{x,t} = 3e^{-4\psq t}\sin(2\pi x)}
$$
\\[0.5cm]
\NI (2) $\displaystyle \frac{\partial^2 u}{\partial t^2}(x,t) = c^2 \frac{\partial^2 u}{\partial x^2}(x,t) \; ; 0<x<1, t > 0, $

$\hspace{3cm} \displaystyle \Un{}{x,0} = 0, \Un{t}{x,0} = 0,  \Un{}{0,t} = 0,  \Un{}{1,t} = 0$\\

\NI \underline{\textbf{Solution:}} \\[0.2cm]
$\displaystyle \Un{tt}{x,t} = c^2 \Un{xx}{x,t} + \sin(\pi x) $\\[0.2cm]

\NI Taking the Laplace transform \\[0.2cm]
$\displaystyle
\LT{\Un{tt}{x,t}} = c^2 \LT{\Un{xx}{x,t}} + \LT{\sin(\pi x)} \\[0.1cm]
s^2 \Un{}{x,s} - s\Un{}{x,0} - \Un{t}{x,0} = c^2 \Un{xx}{x,s} + \frac{sin(\pi x)}{s}
$\\[0.2cm]

\NI Applying the initial conditions; $\Un{}{x,0} = 0$ and $\Un{t}{x,0} = 0 $ \\[0.2cm]
$\displaystyle
s^2\Un{}{x,s} - c^2 \Un{xx}{x,s} = \frac{\sin(\pi x)}{s}
$\\[0.2cm]

\NI Re-arranging \\[0.2cm]
$\displaystyle
c^2 \Un{xx}{x,s} - s^2 \Un{}{x,s} = - \frac{\sin(\pi x)}{s}
$\\[2cm]

\NI Solving the Homogenous problem \\[0.2cm]
$\displaystyle
c^{2} \frac{d^2 u}{dx^2}(x,s) - s^2 \Un{}{x,s} = 0
$\\[0.2cm]

\NI The Auxillary equation is given by: \\[0.2cm]
$\displaystyle
c^2 m^2 - s^2 = 0 \Longrightarrow m^2 - \left(\frac{s}{c}\right)^2 = 0 \Longrightarrow m = \pm \frac{s}{c}
$\\[0.2cm]

\NI The homogenous solution is: 
$\; \displaystyle
\Un{}{x,s} = A_1 e^{\frac{sx}{c}} + A_2 e^{-\frac{sx}{c}}
$\\[0.2cm]

\NI Solving the non-homogenous problem by method of Undetermined Coefficient: \\ i.e
$\; \displaystyle
c^2 \frac{d^2 u}{d x^2}(x,s) - s^2 \Un{}{x,s} = - \frac{\sin(\pi x)}{s} - - - - - - - - - - (*)
$\\[0.2cm]

\NI Let: \\[0.2cm]
$ \displaystyle
\Un{}{x,s} = n_1 \sin(\pi x) + n_2 cos(\pi x) - - - - - - - - - - - - - (a) \\[0.1cm]
\Un{x}{x,s} = \pi n_1 \cos(\pi x) - \pi n_2 \sin(\pi x)  - - - - - - - - - - - - (b) \\[0.1cm]
\Un{xx}{x,s} = -\psq n_1 \sin(\pi x) - \psq n_2 \cos(\pi x) - - - - - - - - - - (c) \\[0.1cm]
$\\[0.2cm]

\NI Comparing or Equating the Coefficients of boths sides: \\[0.2cm]
$ \displaystyle
s: \; \; \; D = 0 \\
s^2: \; A + B = 0 \Longrightarrow A = - B  \\
\text{constant}: 1 = A c^2 \psq \Longrightarrow A = \frac{1}{c^2 \psq} \Longrightarrow B = \frac{-1}{c^2 \psq}\\[0.3cm]
\therefore \frac{1}{s\sbracket{s^2 + c^2 \psq}} = \frac{1}{s c^2 \psq} - \frac{s}{(c^2 \psq)(s^2 + c^2 \psq)} = \frac{1}{c^2 \psq} \sbracket{\frac{1}{s} - \frac{s}{s^2 + c^2 \psq}}
$\\[0.2cm]

$ \displaystyle
\Un{}{x,t} = \frac{\sin(\pi x)}{c^2 \psq} \InverseL{\frac{1}{s} - \frac{s}{s^2 + (c\pi)^2}}
$\\[0.2cm]

$$ \displaystyle \mathbf{
\Un{}{x,t} = \frac{\sin(\pi x)}{c^2 \psq}\sbracket{1 - \cos(c\pi t)}
}$$\\[0.2cm]

\NI \underline{\textbf{NEXT:}} \\Solution of non-linear PDE, by the combined Laplace transform and the new Modified Variational Iteration Method. \\[0.5cm]

\newpage
\begin{center}
\bfseries
SOLUTION OF NON-LINEAR PARTIAL DIFFERENTIAL EQUATION BY THE COMBINED LAPLACE TRANSFORM METHOD AND THE NEW MODIFIED VARIATIONAL ITERATION METHOD
\end{center}

\NI Presenting a reliable combined Laplace transform and the new modified varitional Iteration method to solve some non-linear Partial Differential Equations. This method is more efficient and easy to handle non-linear PDEs.\\

\NI Recall, \\[0.2cm]
$\displaystyle \LTb{\frac{\partial f(x,t)}{\partial t}} = sF(x,s) - f(x,0) \\[0.2cm]
\LTb{\frac{\partial^2 f(x,t)}{\partial t^2}} = s^2 F(x,s) - sf(x,0) - \frac{\partial f(x,0)}{\partial t}
$\\[0.2cm]

\NI Where $F(x,s)$ is the Laplace transform of $(x,t)$ [$x$ is considered as a dummy variable and $t$, a parameter].\\[0.2cm]

\NI Illustrating the basic concept of \underline{\textbf{He's Variational Iteration Method}}, we consider the following general differential equations:\\
$$
L\Un{}{x,t} + N\Un{}{x,t} = g(x,t) - - - - - - - - (i)
$$
\begin{center}
with the initial condition, $\Un{}{x,0} = G(x) - - - - (ii)$
\end{center}

\NI Where $L$ is a linear operator of the first Order, $N$ is a non-linear operator and $g(x,t)$ is non-homogenous term. According to Variational  Iteration Method, we can construct a correction functional as follows: 
$$
U_{n+1} = U_n + \int_{0}^{t} \lambda \left[ LU(x,s) + N(x,s) - g(x,s) \right] ds - - - - - (iii)
$$

\NI where $\lambda$ is a Lagrange Multiplier $(\lambda = -1)$, the subscripts $'n'$ denotes the $nth$ approximation, $\bar{U}_n$ is considered as a restricted variation, i.e $\partial\bar{U}_n = 0$.

\NI Equation (iii) is called a \textbf{Correction Functional}\\
Obtaining the Lagrange Multiplier $'\lambda'$ by using Integration by Part of Equation (i), but the Lagrange Multiplier is of the form $\lambda = \lambda(x,t)$.

\NI Then, Laplace transform of equation (iii), then the correction functional will be in the form:\\[0.25cm]
$$
\LT{\Un{n+1}{x,t}} = \LT{\Un{n}{x,t}} + \LT{\int_{0}^{t}\lambda(x,t)\left( L\Un{n}{x,s} + N\Un{n}{x,s} - g(x,s) \right)ds}, \; n \geq 0 \; - - - - - (iv)
$$

\NI Therefore, 
$$
\LT{\Un{n+1}{x,t}} = \LT{\Un{n}{x,t}} + \lambda \sbracket{L\Un{n}{x,t} + N\Un{n}{x,t} - g(x,t)} \; - - - - - - - - - (v)
$$
To find the optimal value of $\lambda(x,t)$, we first take the Variation with respect to $\Un{n}{x,t}$ and in such a case, the integration is basically the single convolution to $t$, and hence, Laplace transform  is appropriate to use. \\[0.2cm]

\NI thus;\\
$\; \displaystyle
\frac{\partial}{\partial U_n}\LT{\Un{n}{x,t}} + \LT{\lambda(x,t)}\frac{\partial}{\partial U_n}\LT{L\Un{n}{x,t} + N\Un{n}{x,t} - g(x,t)} - - - (vi)
$\\[0.3cm]
Equation (vi) becomes,\\[0.2cm]
$\displaystyle
\LT{\partial\Un{n+1}{x,t}} = \LT{\partial U_n(x,t)} + \LT{\lambda (x,t)}\partial \LT{L\Un{n}{x,t}} - - - - - - - (vii)
$\\[0.3cm]
$\displaystyle
\left\{ i.e \; \; \partial N\bar{U}_{n}(x,t) = 0 \text{ and } \partial g(x,t) = 0\right\}
$\\[0.2cm]

\NI We assume that $L$ is a linear first-order Partial Differential Operator in this chapter given by $\displaystyle \frac{\partial}{\partial t}$ then, equation (vii) can be written in the form\\[0.2cm]
$\displaystyle
\LT{\partial\Un{n+1}{x,t}} = \LT{\partial\Un{n}{x,t}} + \LT{\lambda(x,t)}\sbracket{s\Laplace \partial\Un{n}{x,t}}
$\\[0.2cm]
the extreme condition of $\Un{n+1}{x,t}$ requires that $\partial\Un{n+1}{x,t} = 0$\\[0.2cm]\
$\displaystyle
\Longrightarrow 0 = \LT{\partial\Un{n}{x,t}}\left[1 + s\LT{\lambda(x,t)}\right] \\[0.2cm]
\Longrightarrow 1 + s\LT{\lambda(x,t)} = 0 \\[0.2cm]
\Longrightarrow s\LT{\lambda(x,t)} = - 1 \\[0.2cm]
\Longrightarrow \LT{\lambda(x,t)} = \frac{-1}{s}
$\\[0.2cm]

\NI Taking the Laplace Inverse of both sides\\[0.2cm]
$\displaystyle
\lambda(x,t) = \InverseL{\frac{-1}{s}} \\[0.2cm]
\lambda(x,t) = - 1  \\[0.2cm]
$
This implies $\lambda = -1$\\[0.3cm]

\NI Substituting $(\lambda = -1)$ in equation (iii)\\[0.2cm]
$$
U_{n+1} = U_n - \int_{0}^{t} \sbracket{L\Un{n}{x,s} + N\bar{U}_{n}(x,s) - g(x,s)}ds - - - - - - - - - (viii)
$$

\NI The successive approximation `$U_{n+1}$' of the solution `$u$' will be readily obtained by using the determined Lagrange Multiplier and any selective function $U_n$ consequently, the solution is given by: 
$$
\Un{}{x,t} = \lim\limits_{x \rightarrow \infty}\Un{n}{x,t}
$$
\newpage
Also, from equation (i)
$$
i.e\; \;  LU(x,t) + NU(x,t) = g(x,t) - - - - - - - (*)
$$
taking the Laplace transform of both sides, we have
$$
\LT{LU(x,t)} + \LT{NU(x,t)} = \LT{g(x,t)}
$$







\end{document}