\documentclass[12pt]{report}

\usepackage{amsmath, amssymb}
\usepackage{graphicx}

\newcommand{\bt}[1]{\textbf{#1}}
\newcommand{\ubt}[1]{\textbf{\underline{#1}}}
\newcommand{\sps}{\\[0.2cm]}
\newcommand{\spn}[1]{\\[#1cm]}
\newcommand{\refn}[1]{(\ref{#1})}
\newcommand{\refx}[1]{\refn{eq:#1}}
\newcommand{\NI}{\noindent}
\newcommand{\dsp}{\displaystyle}
\newcommand{\sprime}{'}
\newcommand{\dprime}{''}
\newcommand{\tprime}{'''}
\newcommand{\tti}[1]{\textit{#1}}
\newcommand{\laplace}[1]{\mathcal{L}[#1]}
\newcommand{\inverseLaplace}[1]{\mathcal{L}^{-1}[#1]}

\renewcommand{\baselinestretch}{1.4}

\begin{document}
	\begin{titlepage}
		\centering
		\includegraphics[width=3cm]{unilorin_logo}\spn{.8}
		{\LARGE \bt{UNIVERSITY OF ILORIN, KWARA STATE NIGERIA}}\spn{0.5}
		
		\vspace{1cm}
		
		{\LARGE \bt{Department of Mathematics}}\spn{0.8}
		
		\vspace{1.3cm}
		
		{\Huge Project Proposal}\sps 
		\vspace{1cm}
		
		\tti{BY:}\sps
		\vspace{1cm}
		{\Large \bt{ABUBAKAR, Jamal Adeola}}
				\sps 
			\bt{Matric Number:} 17/56GA010
		
		\vspace{2.5cm}
		
		\bt{JUNE, 2021}
	\end{titlepage}

	%%%%%%%%%%%%%%% FIRST TOPIC %%%%%%%%%%%%%%%%%%%%%
	\section*{TOPIC:}
	{\LARGE \bt{Solutions of Systems of First Order Ordinary Differential Equations and Applications to Real Life Problems}}
	
	\section*{INTRODUCTION}
	\subsection*{Background of the study}
	A system of $n$ linear first order differential equations in $n$ unknowns has the general form
	\begin{equation}
		\begin{array}{ccccccccccc}
			x_1\sprime &=& a_{11}x_1 &+ &a_{12}x_2 &+& \cdots &+& a_{1n}x_n &+& g_1 \sps
			x_2\sprime &=& a_{21}x_1 &+ &a_{22}x_2 &+& \cdots &+& a_{2n}x_n &+& g_2 \sps
			x_3\sprime &=& a_{31}x_1 &+ &a_{32}x_2 &+& \cdots &+& a_{3n}x_n &+& g_3 \sps
			 & & \vdots & & \vdots & &  & & \vdots & & \vdots \sps
			 x_n\sprime &=& a_{n1}x_1 &+ &a_{n2}x_2 &+& \cdots &+& a_{nn}x_n &+& g_n
		\end{array}
		\tag{*}
		\label{eq:astk_1}
	\end{equation}
	where $a_{ij}$'s and $g_i$'s are arbitrary function of $t$.\sps
	If ever term $g_i$ is constant zero, then the system is staid to be \bt{homogeneous}. Otherwise, it is a \bt{non-homogeneous} system if even one of the $g$'s is non-zero.\\
	
	\NI The System \refx{astk_1} is most often given in a shorthand format as a matrix vector equation in the form
	\begin{equation*}
		x\sprime = Ax + g
	\end{equation*}
	
	\begin{equation*}
		\begin{bmatrix}
		x_1\sprime \\
		x_2\sprime \\
		x_3\sprime \\
		\vdots\\
		x_n\sprime
		\end{bmatrix}
		=
		\begin{bmatrix}
			a_{11} & a_{12} & \cdots & a_{1n}\\
			a_{21} & a_{22} & \cdots & a_{2n}\\
			a_{31} & a_{32} & \cdots & a_{3n}\\
			\vdots & \vdots & \vdots & \vdots \\
			a_{n1} & a_{n2} & \cdots & a_{nn}\\
		\end{bmatrix}
		\begin{bmatrix}
			x_1 \\
			x_2 \\
			x_3 \\
			\vdots\\
			x_n
		\end{bmatrix}
		+
		\begin{bmatrix}
			g_1 \\
			g_2 \\
			g_3 \\
			\vdots\\
			g_n
		\end{bmatrix}\sps
	\end{equation*}
	
	\NI For a homogeneous system, $g$ is the zero vector. Hence it has the form
	\begin{equation*}
		x\sprime = Ax
	\end{equation*}
	
	\section*{AIMS AND OBJECTIVES OF THE PROJECT}
	The main aim of the research work is to examine the solutions of systems of first order ordinary differential equations and its applications to real life problems.\sps
	The objectives of the study includes
	\begin{enumerate}
		\item To reduce higher order system value problem to the equivalent first system.
		
		\item to use matrix methods involving eigenvalues and eigenvectors
		
		\item to rewrite linear D.E into a system of two equations
		
		\item to use Laplace Transform method to solve system of linear equations.
		
		\item to solve application problems using first order linear D.E
	\end{enumerate}
	
	\section*{METHODOLOGY}
	\subsection*{1.1 Reduction of higher order to system of first order O.D.E}
	Every $n\mbox{-}th$ order linear equation is equivalent to a system of $n$ first order linear equation.\sps
	
	\NI\bt{Example 1:} Reduce $y\tprime - 3y\dprime + 2y\sprime - 6y = 0$
	From the equation above is equivalent to:
	\begin{eqnarray}
		x\sprime_1 &=& x_2\notag\sps
		x\sprime_2 &=& x_3\notag\sps
		x\sprime_3 &=& 6x_1 - 2x_2 + 3x_3\notag
	\end{eqnarray}
	
	\NI\bt{Example 2:}\sps
	$y\dprime + 5y\sprime - 6y = 0$\sps
	\bt{Solution}\sps
	$y\dprime + 5y\sprime - 6y = 0$
	\begin{eqnarray}
		y\sprime_1 &=& y_2\notag\sps
		y\sprime_2 &=& -5y_2 + 6y_1 \notag
	\end{eqnarray}
	
	
	\subsection*{1.2 Using Matrix method involving Eigenvalues and Eigenvectors}
	This method involves using determinant of the form: 
	\begin{equation*}
		\mathrm{det}(A-\lambda I)
	\end{equation*}
	where $\lambda$ is the eigenvalues and $A$ is a matrix, $I$ is the identity element.\sps
	
	\NI\bt{Example 3:} Find te eigen values and corresponding eigenvectors.\sps
	Let
	\begin{equation*}
		A = \begin{bmatrix}
			1 & 2\\
			2 & 1
		\end{bmatrix}
	\end{equation*}
	\bt{Solution:}\sps
	Recall, $A - \lambda I$\sps
	\begin{eqnarray}
		&=& \begin{bmatrix}
			1 & 2\\
			2 & 1
		\end{bmatrix} - \lambda
		\begin{bmatrix}
			1 & 0\\
			0 & 1
		\end{bmatrix} = 
		\begin{bmatrix}
			1 - \lambda & 2\\
			2 & 1 - \lambda
		\end{bmatrix}\notag\spn{1.2}
		\mathrm{det}(A-\lambda I) &=& (1-\lambda)(1-\lambda) - 2 \times 2 = 0\notag\sps
		& & (\lambda - 1)^2 - 4 = 0\notag \sps
		& & (\lambda - 1)^2 = 4\notag\sps
		\lambda &=& 1 \pm 2\notag\sps
		&& \lambda_+ = 3 \quad \mathrm{or}\quad \lambda_- = -1\notag
	\end{eqnarray}
	Therefore the eigenvalues are $\lambda_1=3,$ and $\lambda_2=-1$.\sps
	To find the eigenvectors\sps
	When $\lambda = 3$
	\begin{eqnarray}
		(A - 3I) &=& 0\notag\sps
		\vec{V_1} &=&  \begin{bmatrix}
			1-3 & 2\\
			2 & 1 - (-1)
		\end{bmatrix}
		\begin{bmatrix}
			x_1 \\
			x_2
		\end{bmatrix}
		=
		\begin{bmatrix}
			0\\
			0
		\end{bmatrix}\notag\sps
		&=& \begin{bmatrix}
			-2 & 2\\
			2 & 2
		\end{bmatrix}
		\begin{bmatrix}
			x_1 \\
			x_2
		\end{bmatrix}
		=
		\begin{bmatrix}
			0\\
			0
		\end{bmatrix}\notag
	\end{eqnarray}
	
	$$-2x_1 + 2x_2 = 0$$
	$$x_1 = x_2\sps$$
	\begin{eqnarray}
		\vec{V_1} &=& \begin{bmatrix}
			x_1\\
			x_2
		\end{bmatrix}
		=
		\begin{bmatrix}
			1\\
			1
		\end{bmatrix}\notag
	\end{eqnarray}
	
	\NI When $\lambda = -1$
	\begin{eqnarray}
		(A + I) &=& 0\notag\sps
		\vec{V_2} &=&  \begin{bmatrix}
			1-(-1) & 2\\
			2 & 1 - (-1)
		\end{bmatrix}
		\begin{bmatrix}
			x_1 \\
			x_2
		\end{bmatrix}
		=
		\begin{bmatrix}
			0\\
			0
		\end{bmatrix}\notag\sps
		&=& \begin{bmatrix}
			2 & 2\\
			2 & 2
		\end{bmatrix}
		\begin{bmatrix}
			x_1 \\
			x_2
		\end{bmatrix}
		=
		\begin{bmatrix}
			0\\
			0
		\end{bmatrix}\notag
	\end{eqnarray}
	
	$$2x_1 + 2x_2 = 0$$
	$$2x_1 = -x_2 \implies x_1 = -x_2\sps$$
	\begin{eqnarray}
		\vec{V_2} &=& \begin{bmatrix}
			x_1\\
			x_2
		\end{bmatrix}
		=
		\begin{bmatrix}
			-1\\
			1
		\end{bmatrix}\notag
	\end{eqnarray}
	
	\subsection*{1.3 Laplace Transform methods of solving system of linear equations}
	To solve a differential equation of the form:
	\begin{equation*}
		af\sprime(t) + bf(t) = g(t)
	\end{equation*}
	given than $f(0)=k$ where $a,b$ and $k$ are constants and $g(t)$ is a known expression in $y$.
	
	\vspace{2cm}
	\section*{REFERENCES}
	\begin{description}
		\item Gabriel Nagy, January 18, 2021, Background of the study, Example 3.
		
		\item Prof. Gbamigbola: Lecture notes (2019) example 2 eqn (B)
		
		\item K.A Stroud (2001) 7th Edition
	\end{description}
	
	
	
	\newpage
	%%%%%%%%%%%%%%% SECOND TOPIC %%%%%%%%%%%%%%%%%%%%%
	\section*{TOPIC:}
	{\LARGE \bt{Laplace Transform: Theory, Problems and Solutions}}
	\section*{INTRODUCTION}
	\subsubsection*{Background of the study}
	Laplace transform is yet another operational tool for solving constant coefficients linear differential equations.\sps
	The process of solution consists of three main steps:
	\begin{itemize}
		\item The given "hard" problem is transformed into a "simple" equation
		
		\item This simple equation is solved by purely algebraic manipulations
		
		\item The solution is transformed back to obtain the solution of te given problem
	\end{itemize}
	
	\section*{AIMS AND OBJECTIVES}
	The main aim of the research work is to examine the Laplace Transform theory, problems and solutions.\sps
	The objective of the study are:
	\begin{enumerate}
		\item To determine Laplace Transform
		
		\item To solve te algebraic equation
		
		\item To determine the inverse transform
		
		\item To determine the Laplace Transform a polynomial
		
		\item Solving initial value problem using Laplace Transform
		
		\item To examine the Laplace Transform and of partial fractions
	\end{enumerate}
	
	\section*{Definition}
	The Laplace Transform is defined in the following way:\sps
	Let $f(t)$ be defined for $t \geq 0$. Then the Laplace transform of $f$, which is denoted $\laplace{f(t)}$ or $F(s)$, is defined by te following equation
	\begin{eqnarray}
		\laplace{f(t)} = F(s) &=& \lim\limits_{T\rightarrow \infty}\int_0^T f(t)e^{-st}\mathrm{d}t\notag\sps
		&=&\int_0^\infty f(t)e^{-st}\mathrm{d}t\notag\sps
		&& \qquad \mathrm{for } \; s > 0\notag
	\end{eqnarray}
	
	\section*{METHODOLOGY}
	\subsubsection*{1.0 Finding Laplace Transform}
	We have methods to find $F(s)$ for a given $f(t)$:\sps
	From the definition:
	\begin{equation*}
		\laplace{f(t)} = F(s) \int_0^\infty f(t)e^{-st}\mathrm{d}t
	\end{equation*}
	For
	\begin{eqnarray}
		f(t) = 1 \implies \laplace{1} &=& \int_0^\infty e^{-st}\mathrm{d}t\notag\sps
		&=&\left[-\frac{1}{s}e^{-st}\right]_0^\infty = \frac{1}{s}\notag
	\end{eqnarray}
	For
	\begin{eqnarray}
		f(t) = t, \laplace{t} &=& \int_0^\infty e^{-st}t\mathrm{d}t\notag\sps
		&=&\left[-\frac{1}{s}e^{-st}\right]_0^\infty + \int_0^\infty \frac{1}{s}e^{-st}\mathrm{d}t = \frac{1}{s}\notag
	\end{eqnarray}
	
	\subsection*{1.1 Finding the Inverse Transform Using Partial Fractions}
	Given a function $f(t)$, we denote its Laplace Transform by $\laplace{f} = F$, the Inverse process is written as
	\begin{equation*}
		\inverseLaplace{F} = f
	\end{equation*}
	\vspace{1cm}
	\subsection*{1.2 Solving ODEs and ODE Systems}
	The application of Laplace Transform method is effective for linear ODEs with constant coefficients and for Systems of such ODEs\sps
	\bt{Example 1.0}\sps
	Solve $f\sprime(t) - f(t) = 2 ----- (a)$ where $f(0)=0$\sps
	$\dsp 2\frac{\mathrm{d}y}{\mathrm{d}x} - y = \sin t ----- (b), \quad y(0) = 1$\sps
	
	\vspace{3cm}
	\section*{REFERENCES}
	\begin{description}
		\item C.T.J Dodson, School of Mathematics Manchester University
		
		\item Marcel B. Finan Arkansas Tech University
		
		\item K.A. Stroud, Dexter J. Booth (2001) 7th Edition(example 1.0 (a) )
	\end{description}
	
	
	
	
	
	
	
	
\end{document}