\documentclass[11pt]{report}
\usepackage{amsmath}
\usepackage{amssymb}
\usepackage{tikz}

\newcommand{\sps}{\\[0.2cm]}
\newcommand{\refn}[1]{(\ref{#1})}
\newcommand{\refx}[1]{\refn{eq:#1}}
\newcommand{\bt}[1]{\textbf{#1}}
\newcommand{\NI}{\noindent}
\newcommand{\real}{ \mathbb{R}}

\renewcommand{\contentsname}{Table of Contents}

\begin{document}
	
	%%%%%%%%%%%%%%%%%%%FRONT COVER%%%%%%%%%%%%%%%%%%%
	\addcontentsline{toc}{chapter}{TITLE PAGE}
	\begin{center}
		\Large \bt{COMPLEX ANALYSIS}
	\end{center}

	\hspace{7cm}
	
	\begin{center}
		\textbf{\textit{BY}}
	\end{center}
	
	\hspace{5cm}
	
	\begin{center}
		\large \textbf{ABIFARIN, NAFISAT BUSAYO
			\\
			18/56EB111}
	\end{center}
	
	\hspace{9cm}
	
	\begin{center}
		A PROJECT SUBMITTED TO THE DEPARTMENT OF MATHEMATICS, FACULTY OF PHYSICAL SCIENCES, UNIVERSITY OF ILORIN, ILORIN, KWARA STATE, NIGERIA.
	\end{center}

	\hspace{7cm}
	
	\begin{center}
		IN PARTIAL FULFILLMENT OF REQUIREMENTS FOR THE AWARD OF BACHELOR OF SCIENCE (B. Sc.) DEGREE IN MATHEMATICS.
	\end{center}
	\hspace{5cm}
	\\ \\ 
	\begin{center}
		\textbf{February, 2021}
	\end{center}
	
	\newpage
	%%%%%%%%%%%%%%%%%%%TABLE OF CONTENTS%%%%%%%%%%%%%%%%%%%
	\addcontentsline{toc}{chapter}{TABLE OF CONTENTS}
	\tableofcontents
	
	\newpage
	\pagenumbering{arabic}
	%%%%%%%%%%%%%%%%%%%CHAPTER ONE%%%%%%%%%%%%%%%%%%%
	\chapter{INTRODUCTION}
		
	\section{Complex Analysis}
	Complex analysis, traditionally known as the theory of functions of complex variable is the branch of mathematical analysis that investigates functions of complex numbers together with their derivatives.\\
	
	\NI As a differentiable function of a complex variable is equal to its Taylor Series (that is, it is analytic). Complex analysis is particularly concerned with analytic functions of a complex variable(that is, holomorphic functions).
	
	\section{Historical Background of Complex Analysis}
	\subsection{Introduction}
	In 18th Century, a reaching generalization of analysis was discovered, centred on the imaginary number of $i=\sqrt{-1}$. The name imaginary arises because squares of real numbers are always positive. Positive numbers have two distinct squares roots - one positive and one negative; zero has a single square root which is zero and negative have no real square roots at all. Numbers formed by combining real and imaginary components such as $2+3i$ are said to be complex. 
	
	\subsection{Why do we study Complex Analysis?}
	Complex analysis serves as an effective capstone course for mathematics major and as a stepping stone to the pursuit of higher mathematics in graduate school.\\
	
	\NI Nearly any number we can think of is a real number. So what happens when we square a number and its gives a negative result, normally, this does not happen because:
	\begin{enumerate}
		\item When we square a positive number, we get a positive result.
		\item When we square a negative number we get a positive result also. Example, $-3x - 3 = +9$.
	\end{enumerate}
	
	\NI Moving on to imaginary numbers, the unit for imaginary number is $i$, which is the square root of $-1$
	\begin{eqnarray}
		i=\sqrt{-1}
	\end{eqnarray}
		
	Because when we square $i$, we get -1
	\begin{equation}
		i^2 = -1
	\end{equation}
	Ordinarily, equation of this form $a^2+9=0$ cannot be solved since our $a=\sqrt{-9}$ which does not have a solution. But with complex analysis, it does have a solution.
	\begin{eqnarray}
		a^2+9=0\notag\\
		a^2=-9\notag\\
		a=\sqrt{-9}\notag\\
		\text{Since } i^2=-1\notag\\
		a = \sqrt{9 i^2}\notag\\
		a = \pm 3i\notag
	\end{eqnarray}
	\subsection{Leonhard Euler}
	Leonhard Euler is the 18th-century Swiss mathematician and is among the most successful mathematicians in history.\\
	Euler made important contributions to Complex analysis and he introduced the scientific notation. He discovered what is known now as \textbf{Euler's formular}, that for any real number $\psi$, the complex exponential equation satisfies
	\begin{equation}
		e^{i\psi} = \cos\psi + i\sin\psi
	\end{equation}
	An this has been called the most remarkable formular in mathematics. Euler's identity is a special case of this:
	\begin{equation}
		e^{i\pi} + 1 = 0
	\end{equation}
	This identity is remarkable as it involves $e,\pi,i,1$ and $0$ and they are arguably the five most important constant in mathematics\sps
	
	\begin{center}
		\begin{tikzpicture}
			\draw[->] (0,-3.2)--(0,4);
			\node at (-0.3,3.9) {Im};
			%%%%%%%%%%%%
			\draw[->] (-3.2,0)--(4,0);
			\node at (3.9,-0.3) {Re};
			%%%%%%%%%%%%%
			\draw (0,0) circle(2.8cm);
			%%%%%%%%%%%%%
			\draw (0,0)--(1,2.61)--(1,0);
			\node at (2.24, 2.86) {$e^{i\psi = \cos\psi + i\sin\psi}$};
			%%%%%%%%%%%%%
			\draw (0.2,0.5) .. controls (0.55,0.55) and (0.65,0.2) .. (0.65,0);
			\node at (0.65,0.6) {$\psi$};
			%%%%%%%%%%%%%
			\node at (-0.25, 3) {$i$};
			\node at (-0.15,-0.22) {$0$};
			\node at (0.5, -0.22) {$\cos\psi$};
			\node at (1.58,1.22) {$\sin\psi$};
			\node at (2.96,-0.22) {$1$};
			
		\end{tikzpicture}
	\end{center}

	\subsection{Georg Friedrich Bernhard Riemann}
	He was a German mathematician born in 1826. His contribution to complex analysis include most notably, the introduction of Riemann surfaces breaking new ground in a natural, geometric treatment of complex analysis.
	
	\subsection{Augustin-Louis Cauchy}
	He was a French mathematician born in 1789. He made pioneering contributions to several branches of mathematics. Cauchy is most famous for his single-handed development of complex function theory. The first pivotal theorem proved by Cauchy, now known as Cauchy's Integral Theorem, is the following
	\begin{equation}
		\oint_c f(z)dz=0
	\end{equation}
	where $f(z)$ is a complex valued function on and within the non-self closed curve $c$(contour) lying in the complex plane. The contour integral is taken along the contour $c$. In 1826, Cauchy gave a formal definition of a residue of a function. If the complex-valued function $f(z)$ can be expanded in the neighbourhood of a singularity as
	\begin{equation}
		\underset{z=a}{\text{Res}} f(z) = \lim\limits_{z\rightarrow a}(z-a)f(z) 
	\end{equation}
	In 1831, Cauchy proposed the formula known as Cauchy's integral formula
	\begin{equation}
		f(a)=\frac{1}{2\pi i}\oint_c\frac{f(z)}{z-a}dz
	\end{equation}
	where $f(z)$ is analytic on $c$ and within the region bounded by the contour $c$ and the complex number $a$ is somewhere in this region.\sps
	
	\NI Clearly, the integrand has a simple pole at $z=a$. He also presented the residue theorem.
	\begin{equation}
		\frac{1}{2\pi i}\oint_c f(z)dz = \sum_{k=1}^{n}\underset{z=a_k}{\text{Res}}f(z)
	\end{equation}
	where the sum is over the n poles of $f(z)$ and within the contour $c$. These results of Cauchy's still form the core of complex function theory.
	
	\section{Complex Number}
	A complex number is a number that can be expressed in the form $a+bi$, where $a$ and $b$ are real numbers and $i$ is a symbol called the imaginary unit and satisfying the equation $i^2=-1$. For the complex number $a+bi$, $a$ is called the real part and $b$ is called the imaginary part. The set of complex number is denoted by $\mathbb{C}$\\
	\begin{center}
		\begin{tikzpicture}
			\draw[->] (0,-1)--(0,4.5);
			\node at (-0.4,4.2) {Im};
			%%%%%%%%%%%%%%%%%%%
			\draw[->] (-1,0)--(6.2,0);
			\node at (5.87,-0.4) {Re};
			%%%%%%%%%%%%%%%%%%%%
			\draw[dashed] (0,3.5) -- (4,3.5)-- (4,0);
			%%%%%%%%%%%%%%%%%%%%
			\draw (0,0)-- (4,3.5);
			\node at (5.06, 3.57) {$z = a+bi$};
			%%%%%%%%%%%%%%%%%%%%%
			\node at (-0.18,-0.25) {$0$};
		\end{tikzpicture}
	\end{center}
	Addition, Subtraction and Subtraction of complex numbers can be naturally defined by using the rule $i^2=-1$ combined with the associative, commutative and distributive laws.
	
	\section{Algebraic Properties of Complex Numbers}
	\begin{enumerate}
		\item When $a+ib=0$ and $a,b,c$ are the real numbers, the value of both $a,b=0$ that is $a=0$ and $b=0$\sps
		\bt{Proof}\sps
		$\dsp
		a+ib=0\\
		0+1\cdot 0 = 0
		$\sps
		Hence, the value of $a$ and $b$ is zero. \\
		%%%%%%%%%%%%%%%%%%
		\item When $a,b,c$ are real numbers and $a+ib=c+id$, then $a=c$ and $b=d$\sps
		\bt{Proof}\sps
		When $a,b,c,d$ exist as real numbers, then\\
		$a+ib=c+id$\\
		$a=c+id-ib$\\
		$\therefore a=c$\\
		%%%%%%%%%%%%%%%%%%
		\item For $z_1, z_2$ and $z_3$ complex numbers, the set must be satisfying the associative, commutative and distributive laws\sps
		\bt{Commutative law for multiplication}\\
		$z_1\cdot z_2 = z_2 \cdot z_1$\sps
		\bt{Associative law for multiplication}\\
		$(z_1z_2)z_3 = z_1(z_2z_3) = z_2(z_1z_3)$\sps
		\bt{Commutative law for addition}\\
		$z_1 + z_2 = z_2 + z_1$\sps
		\bt{Associative law for addition}\\
		$z_1+(z_2+z_3) = (z_1+z_2) + z_3$\sps
		\bt{Distributive law}\\
		$z_1(z_2+z_3) = z_1z_2 + z_1z_3$\sps
		%%%%%%%%%%%%%%%%%%%
		\item The product and the sum of two complex conjugate quantities both exist as ''real''\\
		\bt{Proof}\sps
		Let us assume $z=x+iy$ is the complex number where the real values are $x$ and $y$. Accordingly, the conjugate of $Z$ is equal to $\bar{z} = x-iy$\\
		Now,\\
		$(x+iy)(x-iy) = x^2-i^2y^2 = x^2 + y^2 = z\cdot \bar{z}$~(multiplication)\\
		which is real\\
		And,\\
		$x+iy + x-iy = z + \bar{z}$~(Addition)\\
		${}\qquad\qquad\qquad~~=2x$\\
		Hence, multiplication and addition of two complex conjugates are real.\\
		\item If the product and sum of two complex numbers exist as real, then these complex numbers will be conjugate to each other.\sps
		\bt{Proof}\sps
		Let us assume $z_1 = a+ib$ and $z_2 = c+id$, these are the two quantities where the real values are $a,b,c,d$ and $b\neq 0$ and $d\neq 0$ also.\sps
		So by the theory of assumption, $z_1 + z_2 = a+ib + c+id$\\
		$
			z_1+z_2 = a+ib + c +id\sps
			z_1+z_2 = (a+c) + i(b+d)\sps
		$
		Therefore, $b+d=0 \implies d = -b$\sps
		And\sps
		$
			z_1 \cdot z_2 = (a+ib)(c+id)\\
			~~~~~~~~~~~~ = (ac - bd) + i(ad+bc)$ exists for real.\\
		Hence, $ad+bc=0, -ab+bc=0$\\
		Therefore,\\
		$ad+bc=0$ or $-ab + bc =0$ (since $d=-b$)\\
		Thus,\\
		$z_2 = c+id=a+i(-b)=a-ib=\bar{z}$, which verifies that the values of $z_1$ and $z_2$ are conjugates of each other.
	\end{enumerate}

	\newpage
	
	%%%%%%%%%%%%%%%%%%%CHAPTER TWO%%%%%%%%%%%%%%%%%%%
	\chapter{BASIC CONCEPTS IN COMPLEX ANALYSIS}
	\section{Preamble}
	In this chapter, the basic concepts in Complex Analysis shall be discussed.
	
	\section{Function}
	Function is process or a relation that associates each element $x$ of a set $X$, the domain of the function to a single element $y$ of another set $Y$(possibly the same set), the codomain of the function.\\
	
	\NI If the function is called $f$, this relation is denoted by $y=f(x)$ where the element $x$ is the input of the function and $y$ is the value of the function or the image of $x$ by $f$. A function is represented by the se of all pairs $(x,f(x))$, called the graph of the function. Hence, $f:X\rightarrow Y$ is a function such that for $x\in X$, there is a unique element $y\in Y$ such that $(x,y)\in f$.
	\subsection{Types of Functions}
	\begin{enumerate}
		\item \bt{Identity function}\\
		Let $\real$ be the set of real numbers. If the function $f:\real\rightarrow\real$ is defined as $f(x)=y=x$ for $x\in \real$, then the function is known as Identity function. The domain  and the range being $\real$.
		
		\item \bt{Polynomial function}\\
		A polynomial function is defined by $y=a_0 + a_1x + a_2x^2 + \cdots + a_nx^n$, where $n$ is a non-negative integer and $a_0, a_1, a_2,\ldots,a_n \in \real$. The highest power in the expression is the degree of the polynomial function. Polynomial functions are further classified based on their degree.
		\begin{itemize}
			\item \bt{Constant function:} If the degree is zero, polynomial function is a constant function.
			\item \bt{Linear function:} The polynomial function with degree one such as $y=x+1$ or $y=x$ or $y=2x-5$.
		\end{itemize}
	
		\item \bt{Constant function}\\
		If the function $f:\real\rightarrow\real$ is defined as $f(x)=y=c$, for $x\in\real$ and $c$ is a constant in $\real$, then such function is known as constant function. The domain of the function $f$ is $\real$ and its range is a constant $c$.
	
		\item \bt{Rational function}\\
		A rational function is any function which can be represented by a rational fraction, say $\dfrac{f(x)}{g(x)}$ in which numerator $f(x)$ and denominator $g(x)$ are polynomial functions of $x$, where $g(x) \neq 0$.
		
		\item \bt{Modulus function}\\
		The absolute value of any number, $c$ is represented in the form of $|c|$. If any function $f:\real\rightarrow\real$ is defined by $f(x)=|x|$, is known as modulus function.
		
		\item \bt{Greatest Integer function}\\
		If a function $f\real\rightarrow\real$ is defined by $f(x)=[x], x\in X$. It round off to the integer less than the number. Suppose the given interval is in the form of $(m,m+1)$, the value of the greatest integer is $m$ which is an integer.
		
		\item\bt{Quadratic function}\\
		If the degree of the polynomial function is two, then it is a quadratic function. It is expressed as $f(x)=ax^2 + bx + c$, where $a\neq 0$ and $a,b,c$ are constants and $x$ is a variable.
		
		\item\bt{Algebraic function}\\
		A function is called an algebraic function if it can be constructed using algebraic operations such as addition, subtraction, multiplication, division and taking roots.
	\end{enumerate}
	
	\section{Basic Definitions}
	\begin{enumerate}
		\item\bt{Function of Complex variable}\\
		Functions of $(x,y)$ that depend only on the combination $(x+iy)$ are called functions of a complex variable. the function $f(z)$ of a complex variable $z$ denoted by $\omega$ such that $z=x+iy$ where $x$ and $y$ presents the real and imaginary parts.
		
		\item\bt{Limits of Functions}\\
		The limit of a function at a point $''a''$ in its domain (if it exists) is the value that the function approaches as its argument approaches $''a''$. That is $\lim\limits_{x\rightarrow a}f(x) = L$.
		
		\item\bt{Limit of function of a Complex Variable}\\
		Let a function $f$ be defined at all point $z(f(z))$ in some neighborhood of $z_0$. The $f(z)$ is said to have a limit $''L''$ as $z$ approaches $z_0$. i.e $\lim\limits_{z\rightarrow z_0}f(z) =L ~\forall~ \epsilon > 0$, there exist $\delta>0$ such that $|f(z)-L|<\epsilon$ whenever $|z-z_0|<\delta$.
		
		
		\item\bt{Continuity}\\
		Let $f(z)$ be defined and single-valued in a neighborhood of $z_0$, then $f(z)$ is said to be continuous at $z=z_0$ if for every $\epsilon >0$, there exist $\delta > 0$ such that $|f(z)-f(z_0)|<\epsilon$ whenever $|z-z_0|< \delta$.\\
		For a function to be continuous, these conditions must be satisfied
		\begin{enumerate}
			\item $f(z)$ must be defined
			\item $\lim f(z)$ exist at $z\rightarrow z_0$
			\item $\lim\limits_{z\rightarrow z_0}f(z) = f(z_0)$
		\end{enumerate}
		\item\bt{Differentiability}\\
		Let $f(z)$ be singled-valued in some region $\real$ of the z-plane and $z_0$ be any point in the derivative of $f(z)$ and it is defined as
		\begin{equation}
			f'(z) = \frac{df}{dz} = \lim\limits_{\delta z\rightarrow 0 }\frac{f(z_0 + \delta z) - f(z_0)}{\delta z}
		\end{equation}	
		If a function satisfies the property at a point $z_0$, we say that the function is complex differentiable at $z_0$.
		
		\item\bt{Analyticity}\\
		A function $f(z)$ is said to be analytic in a region $\real$ of the complex plane if $f(z)$ has a derivative at each pint of  $\real$ and if $f(z)$ is single valued.\\
		Pints at which a function $f(z)$ is not analytic are called singular points. A necessary condition for $f(z)$ to be analytic is
		\begin{equation}
			\frac{\partial f}{\partial z}=0
			\label{eq:2_2}
		\end{equation}
		Therefore, necessary condition for $f=u+iv$ to be analytic is that $f$ depends only on $z$. In terms of the real and imaginary parts, $u,v$ of $f$, condition \refx{2_2} is equivalent to
		\begin{equation}
			\frac{\partial u}{\partial x}=\frac{\partial v}{\partial y}
			\label{eq:2_3}
		\end{equation}
		\begin{equation}
			\frac{\partial u}{\partial y}=-\frac{\partial 	v}{\partial x}
			\label{eq:2_4}
		\end{equation}
		Equations \refx{2_3} and \refx{2_4} are known as Cauchy-Riemann equations. They are a necessary condition for $f=u+iv$ to be analytic.
	\end{enumerate}
	
\end{document}